%!TEX root = Constructive Alignment for Introductory Programming.tex

\chapter{Conclusion} % (fold)
\label{cha:conclusion}

Constructive alignment has been widely accepted as a valuable framework for improving the quality of teaching and learning in higher education. \cref{cha:background} of this thesis provided a comprehensive reviewed of prior work on applications of constructive alignment, finding that most applications adopted constructivist approaches to content delivery, but retained traditional approaches to assessment based around the use of assignments and exams. This review indicated that none of the reviewed papers had attempted to recreate the ``web of consistency'' reported in Biggs original work.

\cref{cha:guiding_principles} presented twelve principles, nine related to \emph{how} we teach and three related to \emph{what} we teach. These principles became the foundation for the approach to constructive alignment presented in \cref{cha:approach}, which applied constructive learning theories to unit delivery and assessment. Chapters \ref{cha:example_impl} and \ref{cha:supporting} demonstrated the application of the approach from \cref{cha:approach} in the creation of two introductory programming units, and tools and resources to aid in their delivery. These chapters also demonstrated how the principles from \cref{cha:guiding_principles} embedded within the approach, were realised in the teaching and learning activities and resources created.

The iterations from the action research projects were presented in \cref{cha:evaluation}, along with analysis of issues students faced, and analysis of student progress. Analysis shows the development of the approach, and its underlying principles, as a result of the reflective practice embedded within the iterative action research method used. Increasing student numbers in later iterations also helps demonstrate the applicability of the model presented to the teaching of units involving hundreds of students. 

This thesis has demonstrated how the ``web of consistency'' associated with Biggs early work can be recreated through the application of the principles proposed in \cref{cha:guiding_principles}. As discussed in \cref{cha:discussion}, this approach has enabled teaching staff to successfully deliver a range of introductory programming units. The results demonstrated the effective use of all twelve principles, illustrating how they work together to create a positive, student-centred, teaching and learning environment in which students are rewarded for demonstrating depth of understanding, and encouraged to use their imagination and creativity.

As outlined in the introduction, the key contributions of this thesis are:
\begin{itemize}[noitemsep,nolistsep]
	\item A structured literature review of applications of constructive alignment, examining the areas in which this has been applied and strategies used for delivery and assessment.
	\item A set of guiding principles for the development and delivery of units that aim recreate the ``web of consistency'' evident in Biggs' early work on constructive alignment. 
	\item An approach to constructive alignment developed from the guiding principles with strong links to constructivism in both teaching and learning activities and approach to assessment.
	\item An online task tracking tool to help students track their progress on tasks designed to provide them with feedback in units that delay summative assessment until after the teaching period.
	\item Evaluation of the resulting teaching and learning context and tools, that demonstrates how effective assessment criteria can be used to quickly evaluate student learning outcomes.
	\item An introductory programming curriculum designed using the principles of constructive alignment.
	\item An approach to teaching introductory programming, that embodies the identified principles, with guidelines for implementing this approach.
	\item Example implementations of the approach presented, demonstrating its application to teaching a number of introductory programming units.
	\item A concept-based approach to introductory programming, together with supporting resources including a concept-based text, a game development framework, and range of video podcasts.
	\item Evaluation of the student learning outcomes in terms of their ability to met the intended learning outcomes.
\end{itemize}

As stated in \cref{cha:discussion}, this work is just a beginning and we hope to continue to explore the opportunities discussed in \sref{sec:future_work} in future work.

% chapter conclusion (end)