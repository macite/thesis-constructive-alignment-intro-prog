%!TEX root = Constructive Alignment for Introductory Programming.tex

\chapter{Conclusion} % (fold)
\label{cha:conclusion}

\subsection{Principles in Review} % (fold)
\label{sub:principles_in_review}

Principles stated in \cref{cha:guiding_principles} worked together to achieve these outcomes, as outlined in the following list. 
\begin{itemize}[noitemsep,nolistsep]
	\item Constructive learning theories are central, with everything in the system aiming to help students construct appropriate knowledge, to enable them to think and act as experts. (\Pref{itm:construct})
	\item Activities and resources aim to align with stated intended learning outcomes, which students must demonstrate they have achieved. (\Pref{itm:align})
	\item Formative feedback supports learning throughout the teaching period, with summative assessment qualitatively evaluating how well students have achieved learning outcomes at the end of the unit. (\Pref{itm:formative})
	\item Delivery focuses on communicating important aspects, while providing resources to support students as they need additional details. (\Pref{itm:focus})
	\item Communicating high expectations helps to encourage students to excel, and instil in them a sense of achievement when they succeed. (\Pref{itm:expectations})
	\item Actively supporting students efforts helps each student to achieve the their full potential.  (\Pref{itm:support})
	\item Empowering students by allowing them to manage their own learning helps students become life-long learners. (\Pref{itm:theory_y})
	\item Being agile and willing embraces opportunities for improvement. (\Pref{itm:agile})
	\item Reflective practice helps recognise opportunities for improvement, for both teaching staff and students. (\Pref{itm:reflect})
	\item Having a central focus, such as a programming paradigm, helps provide coherence to the intended learning outcomes and guide the delivery strategy for the unit.  (\Pref{itm:paradigm})
	\item Communicating concepts to students helps them build the knowledge they need to act as experts. (\Pref{itm:concepts})
	\item Always using tools as intended, such as authentic use of programming languages, helps to ensure student experiences develop appropriate knowledge. (\Pref{itm:authentic})
\end{itemize}

These principles were embedded in the approach to constructive alignment presented in \cref{cha:approach}


Sage on the stage, guide by the side.

Could a similar experience have been achieved using a different approach to assessment?

Compromise


% subsection principles_in_review (end)


% chapter conclusion (end)