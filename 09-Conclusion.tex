%!TEX root = Constructive Alignment for Introductory Programming.tex

\chapter{Conclusion and Future Work} % (fold)
\label{cha:conclusion}

Constructive alignment has been widely accepted as a valuable framework for improving the quality of teaching and learning in higher education. \cref{cha:background} of this thesis provided a comprehensive reviewed of prior work on applications of constructive alignment, finding that most applications adopted constructivist approaches to content delivery, but retained traditional approaches to assessment based around the use of assignments and exams. This review indicated that none of the reviewed papers had attempted to recreate the ``web of consistency'' reported in Biggs original work.

\cref{cha:guiding_principles} presented twelve principles, nine related to \emph{how} we teach and three related to \emph{what} we teach. These principles became the foundation for the approach to constructive alignment presented in \cref{cha:approach}, which applied constructive learning theories to unit delivery and assessment. Chapters \ref{cha:example_impl} and \ref{cha:supporting} demonstrated the application of the approach from \cref{cha:approach} in the creation of two introductory programming units, and tools and resources to aid in their delivery. These chapters also demonstrated how the principles from \cref{cha:guiding_principles} embedded within the approach, were realised in the teaching and learning activities and resources created.

The iterations from the action research projects were presented in \cref{cha:evaluation}, along with analysis of issues students faced, and analysis of student progress. Analysis shows the development of the approach, and its underlying principles, as a result of the reflective practice embedded within the iterative action research method used. Increasing student numbers in later iterations also helps demonstrate the applicability of the model presented to the teaching of units involving hundreds of students. 

This thesis has demonstrated how the ``web of consistency'' associated with Biggs early work can be recreated through the application of the principles proposed in \cref{cha:guiding_principles}. As discussed in \cref{cha:discussion}, this approach has enabled teaching staff to successfully deliver a range of introductory programming units. The results demonstrated the effective use of all twelve principles, illustrating how they work together to create a positive, student-centred, teaching and learning environment in which students are rewarded for demonstrating depth of understanding, and encouraged to use their imagination and creativity.

As outlined in the introduction, the key contributions of this thesis are:
\begin{itemize}[noitemsep,nolistsep]
	\item A structured literature review of applications of constructive alignment, examining the areas in which this has been applied and strategies used for delivery and assessment.
	\item A set of guiding principles for the development and delivery of units that aim recreate the ``web of consistency'' evident in Biggs' early work on constructive alignment. 
	\item An approach to constructive alignment developed from the guiding principles with strong links to constructivism in both teaching and learning activities and approach to assessment.
	\item An online task tracking tool to help students track their progress on tasks designed to provide them with feedback in units that delay summative assessment until after the teaching period.
	\item Evaluation of the resulting teaching and learning context and tools, that demonstrates how effective assessment criteria can be used to quickly evaluate student learning outcomes.
	\item An introductory programming curriculum designed using the principles of constructive alignment.
	\item An approach to teaching introductory programming, that embodies the identified principles, with guidelines for implementing this approach.
	\item Example implementations of the approach presented, demonstrating its application to teaching a number of introductory programming units.
	\item A concept-based approach to introductory programming, together with supporting resources including a concept-based text, a game development framework, and range of video podcasts.
	\item Evaluation of the student learning outcomes in terms of their ability to met the intended learning outcomes.
\end{itemize}

As stated in \cref{cha:discussion}, this work is just a beginning and we hope to continue to explore the opportunities discussed in \sref{sec:future_work} in future work.

\section{Future Work} % (fold)
\label{sec:future_work}

This thesis represents a start, rather than end, of a range of research opportunities. Having developed the principles and approach across a number of iterations, there is now opportunities to further investigate portfolios collected, to examine ongoing changes in the activities used and their impact on student results, and to explore the wider application of the approach presented.

The structure literature review reported in \cref{cha:background} identified a range of work on constructive alignment. In collating the data for this review it was noticed that there appeared to be little referencing between the analysed papers, however this reference data was not collected and analysed. A further analysis, and reporting, of these connections may help provide a richer understanding of the field.

Work to date have collected hundreds of student portfolios, each of which captures an individuals learning outcomes from their engagement with the teaching and learning environment. While some analysis has been performed on these portfolios there are many other opportunities that can now be exploited. Student reflections, their programs, reports, custom projects, and research reports all provide different opportunities to explore aspects of the teaching and learning environment from student perspectives.

While student portfolios have demonstrated their ability to create programs, it would be very interesting to repeat an experiment similar to that conducted by \citet{McCracken:2001}. The mathematical nature of the exercises should be changed to better reflect the more general nature of the programming units, but otherwise it would be interesting to evaluate students ability to implement a set specification. Where students were unable to get the program working, it would be interesting to extend the experiment to determine the likely cause of the problem, and the extent of help needed for the student to succeed in implementing the program. This would help improve understanding of the limitation of students at the end of their first year of programming, and likely assistance they will need in implementing programs on their own. 

Continued delivery of the introductory programming units discussed in \cref{cha:example_impl} also offers a range of opportunities to examine the approach in more depth. Consistency of portfolio assessment could be examined to determine if there is much variation between different assessors. A more formal evaluation of staff workloads could be carried out by examining the time spent providing feedback, assessing portfolios, and supporting students during the teaching period. Results could then be compared with the amount of time allocated to these tasks by the University to determine if they are comparable to other approaches.

It would also be very interesting to study students within the environment, examining differences between successful and unsuccessful students. This could help identify reasons why students fail in this highly supportive environment, and suggest strategies that could be applied to help them succeed. It is expected that motivation plays a large role in this.

% additional tool support - forums
% greater flexibility - why fixed teaching periods?

Longitudinal studies could examine how students from early portfolio units progress through later units. Does this experience help them succeed at other programming units? Do students feel they learnt more general skills they could apply to a wider range of units, or was the learning entirely focused on programming knowledge? How do portfolio assessed units relate to other units, from the students perspective? What differences can they identify? Would students like to see more portfolio units? Interviewing students at the end of their degree programme could provide a deeper understanding of what is working, and what can be improved with the proposed approach.

In the broader context, future work is needed to trial the approach in other contexts, and other educational institutes. It would be interesting to see how well the approach can be adapted to non-technical units, where there may be additional challenges in identifying how frequent formative feedback can be achieved. Where the principles from \cref{cha:guiding_principles} can be adopted, it is believed that similar positive learning outcomes will be achieved.

While \sref{sec:challenges_for_wider_adoption} outlined some of the challenges believed to faced this wider adoption, a more systematic analysis of peoples responses to the approach would also be enlightening. The approach is believed to be genuinely better than alternatives, but discussions with others have been met with general resistance. Better understanding people's doubts about the approach could help adapt strategies to more effectively encourage people to consider its use.

Once wider adoption has been achieved, it would be interesting to compare learning outcomes from units using constructive alignment with portfolio assessment, from those using more traditional forms of assessment. By examining a range of units, in different contexts, it should be possible to gain some comparative statistics to support the qualitative findings from this work.

% section future_work (end)


% chapter conclusion (end)