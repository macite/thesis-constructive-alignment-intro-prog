%!TEX root = Constructive Alignment for Introductory Programming.tex

\chapter{Constructively Aligned Introductory Programming Curriculum} % (fold)
\label{cha:constructively_aligned_introductory_programming_curriculum}


\section{Background} % (fold)
\label{sec:background}

% section background (end)


\section{Concepts in Introductory Programming} % (fold)
\label{sec:concepts_in_introductory_programming}
\subsection{Intended Learning Outcomes} % (fold)
\label{sub:intended_learning_outcomes}

Introductory Programming will use a procedures-first approach, and focus on the structured programming principles of organising code using \emph{sequence}, \emph{selection} and \emph{repetition}. Students will learn to use functional and modular decomposition to break problems down, and implement solutions using functions and procedures. Data will be managed using arrays and custom data types. Pointers and memory management will be covered. Parameter passing will be covered, including both pass-by-value and pass-by-reference. Weaved through this will be an iterative development process, a focus on writing clear and legible code, and other good programming practices. In addition to writing code, students will learn to read code for the debugging purposes and to demonstrate the ability to interpret other peoples code. All of this is captured in the intended learning outcomes for Introductory Programming:
\begin{enumerate}
	\item Apply code reading and debugging techniques to analyse, interpret, and describe the purpose of program code, and locate within this code errors in syntax, logic, and/or good practice.
	\item Describe the principles of structured programming, relate these to the syntactical elements of the programming language used, and the way programs are developed using this language.
	\item Construct small programs, using the programming languages covered that include the use of arrays, functions and procedures, parameter passing with call by value and call by reference, custom data types, and pointers.
	\item Use modular and functional decomposition to break problems down functionally, represent the resulting structures diagrammatically, and implement the structure in code as functions and procedures.
\end{enumerate}

Object Oriented Programming takes students who have completed Introductory Programming and introduces them to the object oriented programming paradigm. Students will learn about the core principles of object oriented programming, and how these can be used to create object oriented programs. They will develop programs using integrated development environments, including  unit testing tools. Students will be introduced to visual ways of communicating object oriented designs using the Unified Modelling Language \cite{Fowler:2004}, including both class diagrams and sequence diagrams. Design patterns and heuristics will be used to provide students with a means of evaluating the quality of their designs. As with Introductory Programming students will use a iterative development process. The intended learning outcomes for Object Oriented Programming are:
\begin{enumerate}
	\item Explain the principles of the object oriented programming paradigm specifically including abstraction, encapsulation, inheritance and polymorphism, and explain how these principles are used to create object oriented programs.
	\item Design, develop, test, and debug object oriented programs, using an integrated development environment.
	\item Select and use appropriate collection classes, from the language's class library, to manage collections of multiple objects.
	\item Construct appropriate diagrams and textual descriptions to communicate the static structure and dynamic behaviour of an object oriented solution.
	\item Apply accepted good practices related to the construction of object oriented programs.
\end{enumerate}
% subsection intended_learning_outcomes (end)


Pascal \cite{Becker:2002}



% section concepts_in_introductory_programming (end)

\section{A Simplified Taxonomy to Frame Programming Concepts} % (fold)
\label{sec:a_simplified_taxonomy_to_frame_programming_concepts}

% section a_simplified_taxonomy_to_frame_programming_concepts (end)

\section{Introductory Programming, Procedures First} % (fold)
\label{sec:introductory_programming_procedures_first}

% section introductory_programming_procedures_first (end)

\section{Using Multiple Languages to Focus on Concepts} % (fold)
\label{sec:using_multiple_languages_to_focus_on_concepts}

% section using_multiple_languages_to_focus_on_concepts (end)

% chapter constructively_aligned_introductory_programming_curriculum (end)