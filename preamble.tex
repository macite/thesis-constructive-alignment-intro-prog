%!TEX root = /Users/acain/working/Papers/Thesis/Constructive Alignment for Introductory Programming.tex

% Essential packages
%\usepackage{times}
%\usepackage{helvet}

\usepackage{amsmath, amssymb, amsthm, amstext}
\usepackage[font=small, labelfont=bf, margin=20pt]{caption} 
\usepackage[pdftex]{graphicx,color}
%\usepackage[left=1.5in, right=1in, top=1in, bottom=1in, headheight=13.6pt]{geometry}
\usepackage[left=40mm, right=25mm, top=33mm, bottom=33mm, headheight=14pt]{geometry}

\usepackage[authoryear]{natbib}
\bibliographystyle{agsm} %{abbrv}
\usepackage{subfig}
\usepackage{flushend}
\usepackage{rotating}
\usepackage{multirow}
\usepackage{setspace}
\usepackage{cite}
\usepackage{url}

\usepackage {pdfsync}

\renewcommand{\cite}{\citep}

\usepackage{tgbonum} %% Use the extended bookman font

% Optional customisation packages
\usepackage{mathpazo}
\usepackage{algorithm, algpseudocode}

% \usepackage[compact]{titlesec}
% \titlespacing{\section}{0pt}{16pt}{0pt}
% \titlespacing{\subsection}{0pt}{14pt}{0pt}
% \titlespacing{\subsubsection}{0pt}{12pt}{0pt}

% Page layout
\parindent 0pt
\parskip 0.20in
%\parskip 2.0ex
%\renewcommand{\baselinestretch}{1.5}
%\renewcommand{\baselinestretch}{1.33}
\numberwithin{equation}{section}
\renewcommand{\bibname}{References}
\renewcommand{\contentsname}{Contents}
%\pagenumbering{roman}
\footskip 70pt

%------------------------------------------------------------------------- %
% Stuff needed to use PDF figures inside this document
% \usepackage{times}
\usepackage{xspace}
% \usepackage{graphicx}
\graphicspath{{Figures/}}
%\usepackage[pdftex,colorlinks=true,pdfstartview=FitV,
%  linkcolor=black,citecolor=black,urlcolor=black]{hyperref}
%---------------------------------------------------------------------------% 
% Markup macros for proof-reading
\usepackage[normalem]{ulem} % for \sout
\usepackage{xcolor}
\newcommand{\ra}{$\rightarrow$}
\newcommand{\ugh}[1]{\textcolor{red}{\uwave{#1}}} % please rephrase
\newcommand{\ins}[1]{\textcolor{blue}{\uline{#1}}} % please insert
\newcommand{\del}[1]{\textcolor{red}{\sout{#1}}} % please delete
\newcommand{\chg}[2]{\textcolor{red}{\sout{#1}}{\ra}\textcolor{blue}{\uline{#2}}} % please change

%------------------------------------------------------------------------- %
% Put edit comments in a really ugly standout display
\usepackage{ifthen}
\usepackage{amssymb}
\newboolean{showcomments}
\setboolean{showcomments}{true} % TOGGLE true/false to show or hide comments
\ifthenelse{\boolean{showcomments}}
  {\newcommand{\note}[2]{
	\fbox{\bfseries\sffamily\scriptsize#1}
    {\sf\small$\blacktriangleright$\textit{#2}$\blacktriangleleft$}
   }
  }
  {\newcommand{\note}[2]{}
   \newcommand{\version}{}
  }
\newcommand{\here}{\note{***}{CONTINUE HERE}}
\newcommand\ml[1]{\note{ML}{#1}}
\newcommand\rv[1]{\note{RV}{#1}}
\newcommand\todo[1]{\note{TODO}{#1}}
\newcommand\review[1]{\note{REVIEW}{#1}}
\newcommand\jgs[1]{\note{JGS}{#1}}
\newcommand\pbr[1]{\note{PBRANCH}{#1}}
\newcommand{\msystems}{\emph{forty}}
\newcommand{\mversions}{1057}
\newcommand{\mchangedversions}{1017}  %% This will be mversions - msystems
\newcommand{\mclasses}{55000}

%-------------------------------------------------------------------------
\newcommand{\secref}[1]{Section~\ref{sec:#1}}
\newcommand{\figref}[1]{Figure~\ref{fig:#1}}
\newcommand{\chapref}[1]{Chapter~\ref{chapter:#1}}
\newcommand{\appendixref}[1]{Appendix~\ref{chapter:#1}}
\newcommand{\tabref}[1]{Table~\ref{tab:#1}}
\newcommand{\eqnref}[1]{Equation~\ref{eqn:#1}}

%-------------------------------------------------------------------------
\newcommand{\bs}{\symbol{'134}} % backslash
\newcommand{\us}{\symbol{'137}} % underscore
\newcommand{\ie}{\emph{i.e.}\xspace}
\newcommand{\eg}{\emph{e.g.}\xspace}
\newcommand{\etal}{\emph{et al.}\xspace}
%-------------------------------------------------------------------------

\newcommand{\CA}{Constructive Alignment\xspace}
\newcommand{\IP}{Introductory Programming\xspace}
\newcommand{\OBTL}{Outcomes-Based Teaching and Learning\xspace}


%-------------------------------------------------------------------------

% Macros for type dependency graph section
\newcommand{\tg}{$G^T$}
\newcommand{\ind}[1]{$l_{in}({#1})$}
\newcommand{\outd}[1]{$l_{out}({#1})$}
\newcommand{\inid}[1]{$l^{i}_{in}({#1})$}
\newcommand{\ined}[1]{$l^{e}_{in}({#1})$}
\newcommand{\outid}[1]{$l^{i}_{out}({#1})$}
\newcommand{\outed}[1]{$l^{e}_{out}({#1})$}

\newcommand{\defeq}{\stackrel{\textrm{\scriptsize def}}=}
\newcommand{\edge}[2]{\langle{#1},{#2}\rangle}
\newcommand{\card}[1]{\mid\! #1 \!\mid}  %% Will only work in Math mode
%-------------------------------------------------------------------------
% Metrics used
\newcommand{\fanin}{In-Degree Count\xspace}
\newcommand{\fanout}{Out-Degree Count\xspace}
\newcommand{\indeg}{In-Degree\xspace}
\newcommand{\outdeg}{Out-Degree\xspace}
\newcommand{\WMC}{Weighted Method Count\xspace}
\newcommand{\bc}{Branch Count\xspace}
\newcommand{\fcount}{Field Count\xspace}
\newcommand{\mcount}{Method Count\xspace}
\newcommand{\icount}{Interface Count\xspace}
\newcommand{\supcount}{Super Class Count\xspace}
\newcommand{\paracount}{Parameter Count\xspace}
\newcommand{\tycount}{Type Construction Count\xspace}

\newcommand{\OSS}{Open Source Software}
\newcommand{\OSYS}{Open Source Software Systems}
\newcommand{\OO}{object oriented}


% used metrics
\newcommand{\Metric}[1]{\emph{#1} \emph{Count}}
\newcommand{\Gini}{Gini Coefficient}
\newcommand{\Ginis}{Gini Coefficients}

\newcommand{\LIC}{\emph{Load Instruction Count}}
\newcommand{\SIC}{\emph{Store Instruction Count}}
\newcommand{\CC}{\emph{Number of Branches}}
\newcommand{\NOB}{\emph{Number of Branches}}
\newcommand{\IDC}{\emph{In-Degree Count}}
\newcommand{\ODC}{\emph{Out-Degree Count}}
\newcommand{\MC}{\emph{Number of Methods}}
\newcommand{\NOM}{\emph{Number of Methods}}
\newcommand{\PMC}{\emph{Public Method Count}}
\newcommand{\NOF}{\emph{Number of Fields}}
\newcommand{\NOA}{\emph{Number of Attributes}}
\newcommand{\FC}{\emph{Number of Attributes}}
\newcommand{\MIC}{\emph{Fan-Out Count}}
\newcommand{\TCC}{\emph{Type Construction Count}}
\newcommand{\NOD}{\emph{Number of Descendants}}
\newcommand{\NOC}{\emph{Number of Children}}
\newcommand{\DIT}{\emph{Depth of Inheritance Tree}}
\newcommand{\MCC}{\emph{Fan-Out Count}}

\newcommand{\LICs}{LIC}
\newcommand{\SICs}{SIC}
\newcommand{\CCs}{NOB}
\newcommand{\IDCs}{IDC}
\newcommand{\ODCs}{ODC}
\newcommand{\MCs}{NOM}
\newcommand{\PMCs}{PMC}
\newcommand{\FCs}{NOA}
\newcommand{\MICs}{FOC}
\newcommand{\TCCs}{TCC}
\newcommand{\NODs}{NOD}
\newcommand{\NOCs}{NOC}
\newcommand{\DITs}{DIT}
\newcommand{\MCCs}{MCC}

\newcommand{\GC}{\Metric{Getter}}
\newcommand{\SC}{\Metric{Setter}}
\newcommand{\BC}{\Metric{Boxing}}

%% Additional commands to provide for commenting and notes
\newcommand{\rb}[1]{\raisebox{1.5ex}[0pt]{#1}}
\newcommand{\rbp}[1]{\raisebox{1.5ex}[0pt]{\parbox{7cm}{#1}}}
\newcommand{\cl}{\\ \cline{3-4}}



%%%%%%%%%%%%%%%%%%%%%%%%%%%%%%%%%%%%%%%%%%%%%%%%%%%%%%%%%%%%%%%%%%%%%%%%%%%%%
% Customising chapter headings (optional) - see sectsty.pdf
%\usepackage{sectsty}
%\chapterfont{\large\sc\centering}
%\chaptertitlefont{\large\sc\centering}
%\subsubsectionfont{\small\sc\centering}
%\subsubsectionfont{\centering}


%---------------------------------------------------------------------------
% Customise header to show chapter - see fancyhdr.pdf
\usepackage{fancyhdr}
\pagestyle{fancy} % default ``plain''
\fancyhf{} % clear all rules
\rhead{} % default - chapter
\lhead{\nouppercase{\textsc{\leftmark}}} % default - section
\renewcommand{\headrulewidth}{0.4pt} % thin line
%\renewcommand{\footrulewidth}{0.4pt} % thin line
\makeatletter
%\renewcommand{\chaptermark}[1]{\markboth{\textsc{\@chapapp} \thechapter: #1}{}}
\renewcommand{\chaptermark}[1]{\markboth{\MakeUppercase{\chaptername \xspace \thechapter. #1}}{}}
\cfoot{} % remove the page number in the middle of footer
%\rfoot{\thepage}
\fancyfoot[R]{\thepage}
%\fancyhead[R]{\thepage}

%% The following commands force page numbers to right at start of new chapter
%% Needed because Latex will override fancy headers on first page of new chapters
\fancypagestyle{plain}{\fancyhf{} % clear all header and footer fields 
%\fancyhead[R]{\thepage}
%\renewcommand{\chaptermark}[1]{\markboth{\MakeUppercase{\chaptername \xspace \thechapter. #1}}{}}
%\renewcommand{\headrulewidth}{0.4pt} % thin line
\renewcommand{\headrulewidth}{0pt} 
\renewcommand{\footrulewidth}{0pt}
\fancyfoot[R]{\thepage}} % page number in centre

\makeatother

% Remove the Warning: \headheight is too small (12.0pt):
\setlength{\headheight}{15pt}



%-----------------------------------------------------------------------
% red and blue boxes for drafting notes
%-----------------------------------------------------------------------
\definecolor{boxgray}{gray}{0.9}

\newcommand{\mycolorbox}[3]{
 \vspace{5mm}
 \noindent
 \fcolorbox{#1}{#2}{
   \begin{minipage}[l]{0.95\textwidth}
     \color{#1}
     #3
   \end{minipage}
 }
 \vspace{3mm}
}

\newcommand{\myredbox}[1]{\mycolorbox{red}{boxgray}{#1}}
\newcommand{\mybluebox}[1]{\mycolorbox{blue}{boxgray}{#1}}
\newcommand{\mygreybox}[1]{\mycolorbox{black}{boxgray}{#1}}
\newcommand{\mywhitebox}[1]{\mycolorbox{black}{white}{#1}}



%\usepackage[pdftitle={title here},pdfpagemode=UseOutlines,
%colorlinks=true,pdfauthor={author here},pdfpagetransition=Dissolve,
%bookmarks=true,pdftex=true,plainpages=false,pdfpagelabels,backref]{hyperref}

%PDF hyper-linking (set colors to black for printing)
\usepackage[colorlinks]{hyperref}
\hypersetup{colorlinks,
	linkcolor=black,
	filecolor=black,
	urlcolor=black,
	citecolor=black}


%\usepackage[ps2pdf=true,colorlinks]{hyperref}
%\usepackage[figure,table]{hypcap}
%\hypersetup{
%	bookmarksnumbered,
%	pdfstartview={FitH},
%	citecolor={blue},
%	linkcolor={red},
%	urlcolor={black},
%	pdfpagemode={UseOutlines}
%}
%\makeatletter
%\newcommand\org@hypertarget{}
%\let\org@hypertarget\hypertarget
%\renewcommand\hypertarget[2]{%
%  \Hy@raisedlink{\org@hypertarget{#1}{}}#2%
%} 
%\makeatother 

%\usepackage[T1]{fontenc}
%\usepackage{lmodern}
%\usepackage{times}
%\usepackage{bookman}
%% Use the extended bookman font
%\usepackage{tgbonum}
%\usepackage{cmbright}

%\usepackage{fontspec} 

% DOCUMENT LAYOUT
%\usepackage{geometry} 
%\geometry{a4paper, textwidth=5.5in, textheight=8.5in, marginparsep=7pt, marginparwidth=.6in}
%\setlength\parindent{0in}

% FONTS
% converts LaTeX specials (``quotes'' --- dashes etc.) to unicode
%\defaultfontfeatures{Mapping=tex-text} 
%\setromanfont [Ligatures={Common}, Numbers={OldStyle}]{Hoefler Text}
%\setmonofont{Monaco} 
%\setsansfont{Optima Regular}
%\setromanfont{Optima}

%\setmonofont[Scale=0.9]{Monaco} 
%\setsansfont[Scale=0.9]{Optima Regular}
%\setromanfont{Optima}

\usepackage{listings}
\usepackage{color}
\usepackage{textcomp}
%\definecolor{listinggray}{gray}{0.9}
%\definecolor{lbcolor}{rgb}{0.9,0.9,0.9}
\lstset{
    %float=p,
    %floatplacement=p,
    %boxpos=t,
%	backgroundcolor=\color{lbcolor},
	tabsize=2,
%	rulecolor=,
	language=Java,
    basicstyle=\scriptsize,
    %basicstyle=\footnotesize,
    upquote=true,
    aboveskip={\bigskipamount},
	belowskip={\medskipamount},
	lineskip={\smallskipamount},
    columns=fixed,
    showstringspaces=false,
    breaklines=true,
	numbers=left,
	captionpos=b,
	abovecaptionskip={\bigskipamount},
	belowcaptionskip={\medskipamount},
%    prebreak = \raisebox{0ex}[0ex][0ex]{\ensuremath{\hookleftarrow}},
    frame=single,
    identifierstyle=\ttfamily,
    keywordstyle=\color[rgb]{0,0,1},
    commentstyle=\color[rgb]{0.133,0.545,0.133}\bfseries,
    stringstyle=\color[rgb]{0.627,0.126,0.941}
%	stringstyle=\ttfamily
}


\ifpdf
\DeclareGraphicsExtensions{.pdf, .jpg, .tif}
\else
\DeclareGraphicsExtensions{.eps, .jpg}
\fi