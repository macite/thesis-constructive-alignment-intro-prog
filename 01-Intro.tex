%!TEX root = Constructive Alignment for Introductory Programming.tex

\chapter{Introduction} % (fold)
\label{cha:introduction}

% Context, general problem, current solutions, why these solutions are still lacking, broadly what needs to be solved, my focus and research goal (something like that)

Programming is a critical skill in Computer Science and Software Engineering, as a consequence students in these fields are taught programming from the start of their degree programmes at many Universities. Although we have been teaching programming for a number of decades, learning programming remains challenging \cite{Jenkins:2002,Lahtinen:2005,Lister:2004,McCracken:2001,Ragonis:2007,Robins:2003,Rountree:2002,Renumol:2010,Wiedenbeck:2005}, with a general consensus that many students find programming hard. These challenges, along with the current lack of success in this critical area, have been recognised by \citet{McGettrick:2005} as one of their seven grand challenges in computing education. Despite persistent efforts over many years, we are still a long way from having specific guidance on how best to approach teaching introductory programming. 

Research into teaching introductory programming is an active part of the computing education research field. In their survey of literature on introductory programming \citet{Pears:2007} identified four main categories: curricula, pedagogy, language choice, and tool support. Work on curricula has examined how introductory programming fits into the wider university computing curricula, including recommendations for computing curricula by major professional computing societies \cite{CC2001,CC2008,CSC2013}. Computing education research on pedagogy examines the teaching and learning of introductory programming and include work such as approaches to adapt learning theory for computer science \cite{BenAri:2001}, various views of programming as mathematical \cite{Denning:1989,Dijkstra:1989,Hoare:1969}, problem solving \cite{Palumbo:1990}, or from language syntax and features \cite{Robins:2003}, to work on appropriate cognitive structures \cite{Eckerdal:2005,Green:1996,Green:2000,Soloway:1986}. Language and, by association, paradigm choice is also widely studied, with various papers on the programming language to use in teaching introductory programming \cite{Anik:2011,Boszormenyi:1998,Bishop:2006,Brilliant:1996,Howell:2003,Kelleher:2005,Koffman:1988,Maloney:2010,Mannila:2006,Mannila:2006a,Mody:1991,Pendergast:2006,Roberts:1993} to debates on which programming paradigm should be used early in the curricula \cite{Astrachan:2005,Bennedsen:2004,Cooper:2003,Ehlert:2009,Howe:2004,Lister:2006a,Pattis:1993,Reges:2006}. While research on tool support examines the use of software tools specifically designed to support the needs of novice programmers, including work on automated assessment \cite{AlaMutk:2007,Douce:2005}, visualisation \cite{Naps:2002}, programming environments \cite{Gross:2005,Kelleher:2005,Kolling:2003}.

Advancements from general education literature also provide additional advice on underlying theories and practices. This includes works on the scholarship of teaching and learning \cite{Boyer:1990}, approaches to teaching \cite{Martin:2000}, approaches to learning \cite{Marton:1976a,Entwistle:1991,Trigwell:1991,Trigwell:1999,Marton:2005}, and analysis of the learner's experience \cite{Marton:1997}. While seen as beneficial, many of these general education theories need further research, and a greater inclusion in the computing education research discourse in order to determine their effectiveness in relation to teaching introductory programming.

In their study of students' experiences learning to program, \citet{Bruce:2003} categorised the way students engage with learning to program into five categories range from approaches focusing on ``\emph{getting through the unit}'', to approaches aimed at discovering what it means to be a programmer. Each of these categories can be broadly classified as either a surface or deep approach to learning \cite{Marton:1976a,Ramsden:1992} to program. When engaging surface approaches, students attempt to address the outcomes with as little effort as possible. This leads to situations in which students are primarily motivated by fear of failing, and experience the learning as a struggle, with the topic appearing tedious, hard and boring, adjectives that are too often associated with learning to program \cite{McGettrick:2005}. On the other hand, when engaging deep approaches to learning students seek meaning in what they do, and relate their learning to the bigger picture. To succeed at learning to program, students need to engage deep approaches to learning, as surface approaches alone are unlikely to be sufficient \cite{Bruce:2003}. Pedagogy that encourage students to adopt deep approaches to learning should, therefore, help address some of the issues related to this challenging topic.

Biggs' model of constructive alignment \cite{Biggs:1996c,Biggs:2007}, based upon constructive learning theory (constructivism) and aligned curriculum, aims to enhance student learning outcomes by focusing on \emph{what the student does}. Constructive alignment aims to encourage students to use \emph{deep}, rather than \emph{surface}, approaches to learning. The focus on the central role of the learner in building meaning is derived from constructivist learning theories \cite{Piaget:1950,Phillips:1995,Steffe:1995,Jonassen:1991,Vrasidas:2000}, whilst the alignment of assessment, teaching, and learning activities, has its foundation in instructional design literature \cite{Tyler:1969,Cohen:1987,Ramsden:1992}. Biggs' model is student focused, with clear and intentional alignment of assessment, teaching and learning activities, and unit objectives.

The principles of constructive alignment were discovered through the use of portfolio assessment \cite{Biggs:1996c}. In a unit on psychology, students had been presented with a set of intended learning outcomes, and asked to construct a body of work that demonstrated they had met the stated outcomes by the end of the unit. The resulting environment encouraged and rewarded students for focusing on developing deep understanding of concepts related to the stated outcomes. This original work provided a clear vision, with strong compelling arguments for incorporating student-centred approaches to unit delivery and assessment.

While there is extensive literature on constructive alignment in the general education literature, the reported work has generally focused on adapting unit delivery without changing assessment approaches. Given the central role of assessment in defining curriculum, from the students perspective \cite{Ramsden:1992}, these approaches have not been able to report the same degree of success as reported by \citet{Biggs:1996c}.   

Constructive alignment has also received little attention in computing eduction research related to teaching introductory programming. The work of \citet{Thota:2010} and \citet{Gaspar:2012} have each discussed the principles of constructive alignment in relation to teaching introductory programming. \citet{Gaspar:2012} used constructive alignment to suggest a range of potential changes, while \citet{Thota:2010} provided a deeper discussion on adjusting delivery methods. While these approaches indicate some potential, neither appears able to recreate all aspects of the positive student-centred learning environment reported in the original work on constructive alignment.

Given this context, research that aims to recreate the positive student-centred learning environment reported by \citet{Biggs:1996c} could help inform both computing education, and the wider education research literature. A supportive student-centred learning environment that encourages students to adopt deep approaches to learning could help address some of the challenges associated with teaching introductory programming. At the same time, recreating the student-centred learning environment reported by \citet{Biggs:1996c} could help identify additional principles not currently promoted in associated with constructive alignment.

\section{Research Goals} % (fold)
\label{sec:research_goals}

The primary focus of this research was to improve student learning outcomes in introductory programming units through the application of constructive alignment, as originally proposed by \citet{Biggs:1996c}. As a result, this thesis presents work that recreated the original work on \emph{constructive alignment} to aid in the teaching of \emph{introductory programming}. The goal was to create a supportive, student-centred, learning environment in which students were encouraged and rewarded for engaging in deep approaches to learning, and to identify principles and guidelines that can be used by others looking to create similar environments.

Introductory programming units typically involve hundreds of students. Given the small class size associated with Biggs' original work on constructive alignment, it was necessary to determine a process that enabled the approach to scale to larger class sizes. Improving the scalability of the general approach described by Biggs, the work in this thesis broadens the possible applications for units using constructive alignment with portfolio assessment.

The work in this thesis also examined the environment resulting from the application of the proposed approach. The resulting environment differed from traditional units in higher education, in both its method of delivery and assessment. As a result, one aim of this work was to gain a better understanding of how students learn in this new environment. Understanding issues students face with the new approach will help ensure appropriate support is provided.

A further goals of this work was to develop tools and resources to support the approach, and to examine the use of these tools within the delivery of introductory programming units. These resources helped support the identified principles and approach, which in turn aided student learning.

% section research_goals (end)


\section{Research Approach} % (fold)
\label{sec:research_approach}

The goals of this research work focused on improving student learning outcomes, indicating the practical and applied nature of the research. As a result, this study used a Practical Action Research \cite{Creswell:2008} design based on Mills' \cite{Mills:2010} \emph{dialectic action research spiral}. The model, and identification of its underlying principles, developed over a number of iterations, with each iteration involving a number of steps. The approach involved reflective practice, with each iteration providing insights that feed into subsequent iterations.

The Practical Action Research method used involved iteratively performing the following steps: (1) identify an area of focus, (2) collect data, (3) analyse and interpret the data, and (4) develop an action plan. Each iteration aligned to a teaching period and collected data from students undertaking units using the proposed approach. At the conclusion of the teaching period the data was analysed and an action plan developed for subsequent iterations. The focus of each iteration varied as different aspects of the model required attention.

The final model, as presented in this thesis, is the result of nine iterations of this Practical Action Research method. Each iteration helped refine the approach, with the model stabilising after significant change in early iterations. 

% section research_approach (end)

\section{Key Contributions} % (fold)
\label{sec:key_contributions}

The contributions of this thesis relate to the dual fields in computing education research: education, and computer science and software engineering. This thesis makes the following contributions to the field of education:
\begin{itemize}[noitemsep,nolistsep]
	\item A structured literature review of applications of constructive alignment.
	\item A set of guiding principles for the development and delivery of units that aim recreate the ``web of consistency'' evident in the early work on constructive alignment. 
	\item An approach to constructive alignment developed from the guiding principles with strong links to constructivism in both teaching and learning activities, and assessment.
	\item An online task tracking tool to help students track their progress on tasks designed to provide them with feedback.
	\item Evaluation of the resulting teaching and learning context, and tools, from an educational perspective.
\end{itemize}

This thesis makes the following contributions to computer science and software engineering:
\begin{itemize}[noitemsep,nolistsep]
	\item An introductory programming curriculum designed using the principles of constructive alignment.
	\item An approach to teaching introductory programming, that embodies the identified principles, with guidelines for implementing this approach.
	\item Example implementations of the approach presented, demonstrating its application to teaching a number of introductory programming units.
	\item A concept-based approach to introductory programming, together with supporting resources including a concept-based text, a game development framework, and range of video podcasts.
	\item Evaluation of the student learning outcomes in terms of their ability to met the intended learning outcomes.
\end{itemize}

As a whole, these contributions support \citet{Biggs:1996c} model of constructive alignment, and demonstrate a learning context that is aptly captured in the following quote \citep{Biggs:2007} (p54):
\begin{quote}
	All components in the system address the same agenda and support each other. The students are `entrapped' in this web of consistency, optimizing the likelihood that they will engage the appropriate learning activities \ldots but leaving them free to construct their knowledge their way.
\end{quote}

% section key_findings (end)


\section{Thesis Structure} % (fold)
\label{sec:thesis_structure}

This thesis first considers the existing applications of constructive alignment, and then goes on to develop a model of constructive alignment using portfolio assessment. Following this a curriculum for introductory programming is proposed, evaluated, and discussed in detail. The remainder of this section describes each of these areas in more detail. 

\textbf{\cref{cha:background} - \cnref{cha:background}} provides an extensive literature review of applications of constructive alignment. The main finding of this work indicates that applications of constructive alignment reported in the research literature tend to focus on aligned curriculum, with constructivism being weakly applied, if at all. Staff aligned teaching and learning activities to intended learning outcomes and assessment. Constructive learning theories, when addressed, related to the design of teaching and learning activities, but not to assessment approach.

\textbf{\cref{cha:guiding_principles} - \cnref{cha:guiding_principles}} outlines twelve principles that underlie this work; nine relate to \emph{how} the teaching and learning environment should operate, the remaining three relate to \emph{what} is to be taught. These principles guided decision making throughout this research.

\textbf{\cref{cha:approach} - \cnref{cha:approach}} presents an approach to constructive alignment with strong links to constructivism in both teaching and learning activities, and assessment. While the approach presented was developed for the teaching of introductory programming, general methods for its adoption are discussed. This work helps address the gap identified in the literature review.

\textbf{\cref{cha:example_impl} - \cnref{cha:example_impl}} proposes an introductory programming curriculum designed using the principles of constructive alignment, with strong emphasis on both constructivism and aligned curriculum. This curriculum revives the procedures first approach to introductory programming, and moves the focus from syntax to underlying concepts and abstractions. Through the use of graphical programming grammars, the curriculum incrementally introduces students to programming concepts and then associated syntax. This approach is based on constructive learning theories and aims to help students build viable models of the underlying machine and programming abstractions. 
 
\textbf{\cref{cha:supporting} - \cnref{cha:supporting}} describes a range of resources used to support the approach in teaching introductory programming: an online task tracking tool, game development framework, programming text, and video podcast series. The task tracking tool provided students with details of their progress as they worked throughout the delivery of the unit. The  game development framework presented was designed primarily as a teaching tool to support the procedures-first concept-based approach to teaching introductory programming used in the example units. This is further supported by a programming textbook, and series of video podcasts, that provide further support for this approach to teaching introductory programming. 

\textbf{\cref{cha:evaluation} - \cnref{cha:evaluation}} describes the iterations of the action research method, and provides an evaluation of the teaching and learning context created through the implementation of the approach presented. This work examines the resulting learning environment from an education and computer science and software engineering perspective. The learning environment was rated highly by students, and there was evidence that most students had engaged deeply with the teaching and learning activities. The context was able to scalable to hundreds of students, with academics needing to spend little time on administration and summative assessment, once the system had been put in place. In terms of student outcomes, grades were at least as good as previous approaches but with the advantage that these results are more clearly aligned to the intended learning outcomes. The importance of this alignment, in addition to its educational benefits, is now paramount, as accreditation bodies look to verify that students have met the stated unit and course learning outcomes. 

% Examination of student work showed that students were able to implement supplied designs, with higher grades indicating the ability to design and implement programs. The focus of the curriculum on concepts over syntax is highlighted in the issues raised in student reflections. The issues from these reflections related primarily to learning in general, with fewer issues on programming topics. When students did comment on issues related to programming they the majority were related to understanding concepts, with few students raising issues directly related to syntax in their reflections. Overall, students were able to demonstrate the intended learning outcomes including explaining programming concepts, using programming grammars to learn languages, and the use of these languages to develop programs.

% An analysis of staff time spent on summative and formative assessment indicated that, with this approach, the majority of assessment time is spent on formative tasks that directly support student learning. At the same time, the summative assessment is able to be carried out quickly and efficiently using feedback generated from the frequent formative feedback. This results in staff being able to focus on helping students develop the intended learning outcomes.

\textbf{\cref{cha:discussion} - \cnref{cha:discussion}} elaborates on the implications of the earlier findings, and how these can apply to wider contexts. This discusses the importance of each of the guiding principles, and the aspects of the approach, in crafting the supportive teaching and learning environment. The overall outcomes experienced are discussed along with challenges for wider adoption of this approach. This chapter concludes with a discussion of future work aimed at extending the approach and curriculum presented.

\textbf{\cref{cha:conclusion} - \cnref{cha:conclusion}} provides a summary of the thesis, and argues that the findings presented can aid in the design and delivery of teaching and learning in higher education. 

\textbf{The Appendix} contains data gathered as part of the background literature review.

% section thesis_structure (end)


% chapter introduction (end)

