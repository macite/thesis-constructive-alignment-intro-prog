%!TEX root = Constructive Alignment for Introductory Programming.tex

\chapter{Discussion} % (fold)
\label{cha:discussion}

\graphicspath{{Figures/Discussion/}}

\cref{cha:guiding_principles} presented twelve principles that extend the core principle of constructive alignment to help create a student-centred learning environment centred around constructive learning theories. These principles were then used to guide the creation of the model of constructive alignment for introductory programming presented in \cref{cha:approach}. Chapters \ref{cha:example_impl} and \ref{cha:supporting} provided details of two example units implemented using this approach, and the supporting resources used to assist students in constructing appropriate knowledge. With \cref{cha:evaluation} providing an analysis of the evolution of the approach and assessment criteria, through iterative action research, and evaluations of issues students faced and the use of the Doubtfire tool for tracking student progress. 

This chapter discusses the overall experience of developing and delivering units using this approach. \sref{sec:general_applicability_of_constructive_alignment} provides support for Biggs' claim that for the general applicability of constructive alignment, providing some illustrations of how other units could be implemented using the approach from \cref{cha:approach}. 

% 4: Risk Assessment of Approach

% How does the approach address risks related to "Surface Learning and Plagiarism" (subsection 1) and "Staff and Student Workloads" (subsection 2).

% 5: Challenges for Wider Adoption

% What is likely to limit wider adoption of this approach? Discuss what we believe is the reason why people will resist this.
% - Required change in view of education (shift to predominant constructivist view)
% - Perceived release of control
% - Perceived workload issues
% - Required Expertise
% - Misunderstanding required changes  (see section related to removing principles)

% 6: Overall Outcomes

% Is it all worth it?
% What changes have been observed?

% - Supportive environment focused on learning
% - Students are empowered to engage their creativity and imagination
% - Widening of perspective - support students in a wider range of activities
% - Supporting a wider range of capabilities
% -- No upper bound on what can be learnt (and given credit for)
% -- While still catering for 
% - Encourages interaction between staff and students -- interviews give you access to the best students (and gets them talking to us)

% 7: Other Points of Interest

% Some smaller points of discussion:
% - Use of Pascal to remove focus from syntax (support existing work)
% - Objects First / Objects Later and the Paradigm Shift
% - Failing Students and Student Motivation (link to Guzdial work on non-majors)
% - Success of Resources -- short/specific had longer life, separation of resources useful

% 8: Future Work

% - Application to non-technical units and beyond tertiary institutes
% - Find out why people do resist this, and ways to help educators transition.
% - With wider adoption perform comparison between approaches (traditional/CA portfolio)
% - Deeper analysis of portfolio contents
% - Redo McCracken study -- can they program at the end of this?



\section{General Applicability of Constructive Alignment} % (fold)
\label{sec:general_applicability_of_constructive_alignment}

Biggs' original proposal of constructive alignment concluded with the following question:

\begin{quote}
	``Can the principle of constructive alignment be generalised from the context of in-service teacher education?'' \citet{Biggs:1996c}
\end{quote}

While others have applied the core principle of constructive alignment, as discussed in \cref{cha:background}, this work has identified additional principles to help recreate the ``web of consistency'' that provided the inspiration for \citet{Biggs:1996c,Biggs:1999} constructive alignment. The twelve principles stated in \cref{cha:guiding_principles} underpin the approach to constructive alignment described in \cref{cha:approach} which, together with the principles, guided the design, development, and delivery of the units described in \cref{cha:example_impl}. The results? A supportive, student-centred, teaching and learning environment in which, to use the words of \citet{Biggs:2007} (p.51), students consistently ``\emph{stun}'' teaching staff with the ``\emph{rich and exciting}'' work they demonstrate in their portfolios.

As outlined in \cite{Biggs:1996c}, the model of constructive alignment makes intuitive sense, and comes together as a whole when the following conditions are met.
\begin{enumerate}[noitemsep,nolistsep]
	\item Teaching staff are clear about the \emph{intended learning outcomes}.
	\item Assessment criteria are provided to indicate how these outcomes can be met at various levels of achievement, forming a hierarchy from barely satisfactory to most acceptable.
	\item Students are required to perform activities that are likely to elicit the required understandings.
	\item Students provide evidence that their learning has matched the stated outcomes.
\end{enumerate}

\cref{cha:approach} demonstrated how the guiding principles described in \cref{cha:guiding_principles} can be applied to create an approach to teaching introductory programming that meet, and in many regards go beyond, these conditions. The processes described started with the clear expression of intended learning outcomes, with the development of assessment criteria providing the required performance objectives required for different grade outcomes. The development of teaching and learning activities aimed to provide students with tasks likely to engage them in activities that will enable them to construct appropriate understandings, and produce evidence they can include to demonstrate their newly gained knowledge. This evidence could then be collected together and presented in student portfolios as a means of demonstrating how the stated objects had been met.

Therefore, the approach presented in \cref{cha:approach}, along with the example implementation discussed in Chapters \ref{cha:example_impl} to \ref{cha:evaluation}, provide additional support for Biggs' claim that constructive alignment using portfolio assessment can be generalised to a range of educational contexts. 

The general applicability of constructive alignment gives rise to the question: Can the approach presented in \cref{cha:approach} be used beyond the context of introductory programming? We believe so. In fact this approach has been used to implement a range of technical units, each with similar positive results, as outlined in the following list.

\begin{itemize}[noitemsep,nolistsep]
	\item Artificial Intelligence for Game used intended learning outcomes related to the use of Artificial Intelligence in creating immersive gaming experiences. Student portfolios included a number of programs to demonstrate various techniques, with higher grades demonstrating the application of learnt concepts in the development of a program of the students own invention.
	\item Concurrent Programming covered the use and implementation of concurrency control mechanisms such as semaphores, barriers, and channels. Portfolios included implementations of these utilities, along with programs demonstrating solutions to classic synchronisation problems.   
	\item Enterprise Software Development involved the use of a range of software tools to implement larger, multi-tier, solutions to business scenarios. Portfolios included demonstrations of various technologies, architectural designs, and technical demonstrations of core components of these designs.
	\item Games Programming introduced concepts related to game design, and the implementation of game engine concepts. Portfolios included demonstrations of various programming techniques and optimisations related to game development, with students implementing game prototypes for higher grades.
	\item Mobile Software Development explored the implementation of software for mobile devices, and associated usability issues. Students applied concepts they learnt in the creation of their own programs for higher grades.
\end{itemize}

In each case the units involved incorporated the principles from \cref{cha:guiding_principles}, with the central role of programming paradigms (\Pref{itm:paradigm}) in the development of introductory programming units being adjusted to focus on key aspects relevant for each unit. The use of portfolio assessment in each case meant that similar, in many cases identical, assessment criteria were able to be used.

While all of these units are technical in nature, we believe the approach can also be applied to non-technical units. The large majority of processes described in \cref{cha:approach}, and their associated guidelines, are applicable for both technical and non-technical units. Assessment criteria is where this is likely to differ. Programming units involve a significant focus on functioning knowledge, involving the practical application of concepts learnt to problem solving and the creation of computer software. The resulting assessment criteria then promote the use of this functioning knowledge in a creative context, and the submission of the resulting artefacts to be eligible for high grades. Where a unit primarily focuses on declarative knowledge such assessment criteria will not be applicable. In these contexts alternative assessment criteria would be required, with the aim of students demonstrating relational levels of understanding in some creative context enabling students to explore aspects of the unit they find most interesting. For example, a unit may require students to undertake wider reading and present their findings to the class or in an extended literature review.

The use of the approach from \cref{cha:approach} would also be particularly well suited to assess learning outcomes for units that include significant use of group work, such as with team-based final year capstone projects. In these unit students work as part of a team, with obvious challenges in assessing the learning outcomes from individual students as final work products are a team effort. Using the approach from \cref{cha:approach} it would be possible to create a learning environment for these units in which:
\begin{itemize}[noitemsep,nolistsep]
	\item Intended learning outcomes capture the required technical and teamwork skills and understandings students needed to demonstrate to successfully complete the unit.
	\item Assessment criteria indicate how these outcomes needs to be demonstrated in order to achieve different grade outcomes.
	\item Students engage in teamwork activities, which are likely to elicit the required outcomes.
	\item Each students collects evidence that they have met all of the intended learning outcomes, aligns their evidence in a Learning Summary Report, and presents this for assessment.
\end{itemize}

In this way, each student's grade would reflect how well they, as individuals, had met the intended learning outcomes. Creating such a scheme would require the embodiment of all principles stated in \cref{cha:guiding_principles}, particularly the need to trust and empower students in their learning (\Pref{itm:theory_y}).

One place where we differ from the recommendations of \citet{Biggs:2007} is in the use of portfolios for larger class sizes. Incorporating frequent formative feedback, tracked by the online Doubtfire tool, made it possible to use portfolio assessment with classes in excess of 300 students (323 students completed the introductory programming units in iteration 8). The frequent formative feedback meant that student work submitted in their portfolios had \emph{already} been checked, possibly multiple times, and if completed successfully this had been indicated in the Doubtfire tool. As a result, the majority of student work did not need to be re-checked in their final portfolios, and grades could be quickly determined. 

Reflections from teaching staff indicate that the resulting process enabled student portfolios to be assessed in significantly less time than it took to assess the previously used exams -- which consisted of multiple choice, short answer, and coding questions. Given this, and ongoing improvements through reflective practice, it is also believed that the use of portfolio assessment could scale to significantly larger class sizes.

% subsection general_applicability_of_constructive_alignement (end)


\clearpage
\section{Approach in Relation to Previous Work} % (fold)
\label{sec:approach_in_relation_to_previous_work}

Previous work on applying constructive alignment, as reported in the structured literature review from \cref{cha:background}, has predominantly seen the application of constructive alignment as simply the task of staff aligning teaching and learning activities with the unit's intended learning outcomes. This differs vastly from the view of constructive alignment presented in this thesis, where constructive alignment is seen as a much greater shift in educators understanding and approach to teaching and learning, one that is centred upon the principles outlined in \cref{cha:guiding_principles}. When constructive alignment is seen in this way, the resulting teaching and learning environment \emph{must} shift from a teaching-centred to a student-centred focus, with all aspects working together to guide and support students in the construction of their own knowledge.

This change in conception requires an adjustment to fundamentally held notions of effective education and its assessment. The continued use of arbitrarily weighted assignments and exams is ineffective in communicating the focus on learning and understanding. Restrictive assessment practices, the result of largely Theory-X dominated views of motivation, limit student opportunities, confining them to pre-set bounds defined as effective learning by teaching staff. By trusting and empowering students through the use of open assessment practices, guided by a Theory-Y view of motivation, these artificial constraints disappear and students are free to use their imagination and creativity. The results, as originally reported by \citet{Biggs:2007} and supported by our experiences outlined in \cref{cha:evaluation}, are portfolios that are truly amazing. 

Prior to adopting the approach outlined in \cref{cha:approach}, the staff involved in teaching the units discussed in \cref{cha:example_impl} often felt that assessment was a negative experience. Marking assignments and exams identified, often for the first time, a large range of student misconceptions. The illusion that lectures had been effective in transferring knowledge to students disappeared, but too late to effect learning outcomes. This was further reinforced with potential arbitrary weights of assignments and exams often resulting in cases where teaching staff felt the student results did not match the knowledge, or lack thereof, that they had demonstrated in the final assessment tasks. It is not surprising that this form of assessment often resulted in disappointment.

All of this changed with the shift in approach. Final student assessment becoming a positive and rewarding experience for teaching staff. Use of frequent formative feedback meant that students misconceptions were addressed often, allowing teaching staff to direct students and guide them to better understand unit concepts. ``Assessments'' were no longer final, so students were encouraged and rewarded for incorporating feedback they received, with each student receiving individual feedback based upon their current level of understanding. Assessment criteria provided a means for teaching staff to set expectations, while still providing opportunities for students to pursue their own interests and to use their imagination and creativity. So final assessment did not hold any \emph{negative} surprises. Portfolios provided an opportunity for students to ``\emph{show off}'' what they had learnt. Where students had achieved Distinction and High Distinction results, these portfolios often went well beyond staff expectations, making assessment a rewarding experience for teaching staff as students demonstrated just how much they had been able to achieve.

It is hard not to draw parallels between these different approaches to education and different software development life-cycle models. Traditional assessment approaches, based upon assignments and exam, can be likened to the Waterfall approach. Teaching and learning activities are delivered in sequence, with little feedback from students. When feedback is provided on summative assignments it is often overlooked \cite{Black:1998}, as students focus on the grade they achieved rather than on opportunities for deeper learning. The accumulated misunderstandings are then ever apparent in the final examination. In contrast, the iterative process outlined in \cref{cha:approach} is more akin to processes in Agile software development. Students and staff interact frequently, enabling staff to guide students with the focus being clearly on their learning during the teaching period. Final summative assessment in student portfolios demonstrate to what level students have been able to achieve stated unit outcomes, with higher grades indicating a demonstration of deeper understanding.

% section approach_in_relation_to_previous_work (end)

\section{Principles and Approach in Review} % (fold)
\label{sec:approach_and_principles_in_review}

\cref{cha:guiding_principles} provided twelve principles used to guide the creation of the student centred approach to teaching introductory programming presented in \cref{cha:approach}. While each principle has its own individual merits, we believe that they are all required to work together in order to achieve the outcomes presented. This section draws upon the experiences of teaching staff, as reported in \cref{cha:evaluation}, to discuss the potential impact of failing to address each of these principles on the overall learning environment. The following list outlines how the principles are designed to work together, as a reference for this discussion.

\begin{itemize}[noitemsep,nolistsep]
	\item Constructive learning theories are central, with everything in the system aiming to help students construct appropriate knowledge, to enable them to think and act as experts. (\Pref{itm:construct})
	\item Activities and resources aim to align with stated intended learning outcomes, which students must demonstrate they have achieved. (\Pref{itm:align})
	\item Formative feedback supports learning throughout the teaching period, with summative assessment qualitatively evaluating how well students have achieved learning outcomes at the end of the unit. (\Pref{itm:formative})
	\item Delivery focuses on communicating important aspects, while providing resources to support students as they need additional details. (\Pref{itm:focus})
	\item Communicating high expectations helps to encourage students to excel, and instil in them a sense of achievement when they succeed. (\Pref{itm:expectations})
	\item Actively supporting students efforts helps each student to achieve the their full potential.  (\Pref{itm:support})
	\item Empowering students by allowing them to manage their own learning helps students become life-long learners. (\Pref{itm:theory_y})
	\item Being agile and willing embraces opportunities for improvement. (\Pref{itm:agile})
	\item Reflective practice helps recognise opportunities for improvement, for both teaching staff and students. (\Pref{itm:reflect})
	\item Having a central focus, such as a programming paradigm, helps provide coherence to the intended learning outcomes and guide the delivery strategy for the unit.  (\Pref{itm:paradigm})
	\item Communicating concepts to students helps them build the knowledge they need to act as experts. (\Pref{itm:concepts})
	\item Always using tools as intended, such as authentic use of programming languages, helps to ensure student experiences develop appropriate knowledge. (\Pref{itm:authentic})
\end{itemize}

\subsection{Constructive Learning Theories} % (fold)
\label{sub:constructive_learning_theories}

\pref{itm:construct} plays a central role in providing the motivation for all of the other principles. Following the mechanisms underlying this approach without adjusting the underlying view of education is likely to be challenging. Where understanding is not seen as being constructed by individual learners there is little need to attempt to create teaching and learning activities that actively engage students. Effective teaching becomes a matter of effectively delivering the required knowledge, resulting in teacher-centred activities coming to the fore. 

Adopting constructive learning theories in teaching and learning activities addresses only part of the overall teaching and learning environment. The central role of assessment implies that these theories need to extend beyond teaching and learning activities to assessment. Applying traditional assignment and exam assessment schemes limits the opportunity for students to demonstrate their understanding in a way that is meaningful to them. Given their role in constructing their knowledge, they are best situated to determine what represents the best demonstration of their understanding. Implementing the other principles without adopting more flexible assessment approaches is likely, therefore, to limit the effectiveness of the learning environment overall as students are limited by the assessment approach.

While the core ideas of constructivism need to be incorporated, \cref{cha:guiding_principles} argued that constructive learning theories needed to be moderated with pragmatic aspects to avoid unproductive discovery learning. While knowledge transmission cannot be achieved, approached based on discovery learning have been shown to be ineffective, as was found in the early iterations discussed in \cref{cha:evaluation}. 

These competing views can be best thought of as a continuum based upon communication; from knowledge transmission with objectivist learning theories, to discover learning with constructivist theories. This continuum is shown visually in \fref{fig:balanced_constructivism}, which depicts the underlying concepts that drive the actions of teaching staff when they hold these views. At the objectivist end of the continuum communication is key, and teaching staff work to communicate all of the aspects students need to understand. This view is teacher-centred as the teacher communicates the required understanding for students to passively absorb. At the other end, the extreme constructivist view discards any value in communication, instead students are placed in situations and asked to ``discover'' the knowledge themselves. Adopting either of these extreme positions is not likely to result in an effective, student-centred, learning environment.

\begin{figure}[htbp]
	\centering
	\includegraphics[width=\textwidth]{BalanceConstructivism}
	\caption{Thoughts that guide teaching staff at either end of the constructivism-objectivism continuum. \cref{cha:guiding_principles} advocates a pragmatic approach to constructivism, somewhere on the constructivist side of the continuum. }
	\label{fig:balanced_constructivism}
\end{figure}

\pref{itm:construct} requires a middle ground approach. To embrace this approach, educators need to accepts the central role of the student in constructing their own knowledge, while also accepting that communication can be a powerful tool to help guide this construction. Communication then becomes a valuable tool, with teaching staff being encouraged to communicate as little or as much as is \emph{needed} by the students at that stage of their learning. Communication is not used to transfer knowledge, but to help guide students so that they may see how viable knowledge can be shaped.

Reflections from the evaluation of the unit deliveries across the various iterations in \cref{cha:evaluation} provide some support for taking this middle ground approach. Teaching staff attribute many of the failings of the object oriented programming unit in Iteration 2 to an overly zealous application of constructive learning theories, and discovery learning. Moving back from these extreme, to a more moderate application of constructive learning theories addressed this situation in later iterations, and as staff expertise in guiding students improved, so did student results.

% subsection constructive_learning_theories (end)

\subsection{Aligned Curriculum} % (fold)
\label{sub:aligned_curriculum}

Aligning all aspects of the teaching and learning environment (\Pref{itm:align}) makes good intuitive sense. Failing to address this principle is likely to result in one of three potential outcomes based upon the alignment between the three component parts: teaching and learning activities, assessment tasks, and intended learning outcomes.

\begin{enumerate}[noitemsep,nolistsep]
	\item Where assessment aligns to intended learning outcomes, but activities do not, students are likely to be unprepared for the assessment tasks. While assessment does define the curriculum for the students, activities provide a means of preparing them for this assessment. 
	\item Where activities align to intended learning outcomes, but assessment does not, students may build appropriate knowledge but the assessment is unlikely to identify this, with final student results failing to represent their ability to meet the intended learning outcomes. 
	\item Where neither assessment nor activities align with intended learning outcomes, students are not likely to learn what was intended in terms of their overall degree programme outcomes. While students may effectively learning something valuable, and the assessment appropriately report this, the misalignment is likely to cause issues for students in later units, or professional life, where they are required to rely upon the intended learning outcomes of the unit.
\end{enumerate}

Alignment is, therefore, a critical aspect in creating an effective learning environment. By aiming to achieve consistency between teaching and learning activities, assessment tasks, and intended learning outcomes, teaching staff provide students with the greatest opportunity to learn the required knowledge, in an effective manner. 

It is also critical to understand that students must also be involved in this alignment process. The interplay between \pref{itm:construct} and \pref{itm:align} means that students, not staff, are in the best position to report on how teaching and learning activities and assessment tasks aligned with the intended learning outcomes. Students followed the planned activities, carried out the assessment tasks all of which helped them in the construction of their knowledge. It is the students, therefore, who truly know how these activities aligned.

The implication of this is that there is not likely to be one measure of alignment for a set of teaching and learning activities. Each students learning will be unique, based upon their prior experiences and current knowledge structures, resulting in the activities influencing each students in a unique manner. Alignment reported by staff in carefully prepared matrices are, therefore, illustrative at best. It should be important for activities to provide students with a range of opportunities to engage with each learning outcome, giving students the best opportunity to actually achieve these learning outcomes. The matrices for the units presented in \cref{cha:example_impl} demonstrate this wide coverage of outcomes, see \tref{tbl:intro_prog_matrix} and \tref{tbl:oop_matrix}. In these units students were provided with a number of opportunities to engage with each of the intended learning outcomes. 

% subsection aligned_curriculum (end)

\subsection{Assessing Learning Outcomes} % (fold)
\label{sub:assessing_learning_outcomes}

\pref{itm:formative} aims to encourage educators to evaluate student learning outcomes in terms of their developed understanding at the end of the teaching period. To achieve this, \cref{cha:guiding_principles} advocates the use of frequent formative feedback and assessment tasks that require students to articulate their understanding, in addition to practical application, of the concepts covered. The following list outlines three ways in which this principle can be violated, each of which is then discussed. 

\begin{enumerate}[noitemsep,nolistsep]
	\item Frequency of feedback can be reduced.
	\item Assessment can focus on product outcomes and not require students to articulate their understanding.
	\item Summative tasks can be used during the teaching period.
\end{enumerate}

There may be some temptation to reduce the frequency of formative feedback, to reduce staff and student workloads. However, rapid iterations are key to ensuring student learning remains ``on track''. Once again, this process can be likened to software development lifecycles. Agile software development processes incorporate frequent client feedback to ensure projects remain on track. Lengthening the time between these iterations, for learning as in software development, results in larger opportunities for loses in productivity due to misunderstandings. As the frequency of formative feedback is reduced, there are less opportunities for staff to positively influence student outcomes, and overall results are likely to be weaker.

This temptation is also ill-founded, as frequent feedback does not aim to increase workloads but to distribute this work more consistently throughout the teaching period. \fref{fig:formative_feedback} illustrates this aim, showing that the goal is to provide smaller frequent feedback in order to maintain similar overall effort. Where this can be achieved, students are more likely to be develop appropriate understandings as misconceptions can be addressed earlier in the process. For the example units presented in \cref{cha:example_impl}, the use of weekly formative feedback helped ensure that each task could be assessed quickly, thereby ensuring students received their feedback in a timely manner. Had larger tasks been used, additional time would be needed to assess these thereby further delaying feedback.

\begin{figure}[htbp]
	\centering
	\includegraphics[width=0.8\textwidth]{FormativeFeedback}
	\caption{Illustration of time allocation to assessment tasks, with rectangle areas representing effort expended by staff providing feedback or students preparing submissions.}
	\label{fig:formative_feedback}
\end{figure}

The second issue listed above relates to the assessment of student understanding, in addition to product outcomes. This is particularly relevant to programming units, where it is easy to assess the programs students create rather than attempting to assess their understanding. Assessing product outcomes alone encourages surface approaches to learning, as it is the product and not the understanding that is being assessed. Including some tasks that require students to articulate their understanding provides opportunities for students to engage appropriate cognitive levels, helping them develop the required understanding, while also communicating the importance of this understanding to the students. In this way, these tasks help to encourage students to appropriately engage with learning activities, and to use deep approaches to learning.

The final issue relates to the use of summative assessment, rather than formative feedback, during the teaching period. Including summative assessment in this manner breaks several critical aspects of the approach presented. Assessing the tasks within the teaching period means that an overall assessment of student learning outcomes is no longer possible. This was experienced in one of the early iterations discussed in \cref{cha:evaluation}, with results failing to match outcomes demonstrated in student portfolios.

Using summative assessment during the teaching period also limits the likelihood of students incorporating feedback they receive. Summative assessment is, by its very nature, final, and so students are not encouraged to learn from this assessment. One of the positive aspects of using formative feedback during the delivery of the example units was that understanding became the key focus. Tasks were not complete until students had demonstrated the required understanding. There was no punishment for not having understood an aspect of a topic on the first attempt, freeing teaching staff to provide relevant feedback and guide students toward the required understandings.

While summative assessment during the teaching period works against the overall principles stated in \cref{cha:guiding_principles}, the distinction between formative feedback and summative assessment changes when using summative assessment that aims to provide a qualitative, holistic, assessment of student outcomes. Constructive learning theories require staff to gain an understanding of the likely level of understanding students have developed in order to provide formative feedback. As a result, teaching staff could potentially provide summative assessment of student performances at any stage during the teaching period. In effect, the assessment at the end of the teaching period represents an arbitrary point in time at which this assessment does occur. Ideally, additional flexibility in education could allow this point to be adjusted for individual students further catering for a wide range of capabilities.

Formative feedback with qualitative, holistic, summative assessment of student outcomes is seen as critical to the success of the approach presented in this thesis.

% subsection assessing_learning_outcomes (end)

\subsection{Supporting Principles} % (fold)
\label{sub:supporting_principles}

Principles \ref{itm:focus} to \ref{itm:reflect} provide underlying support for the adoption of constructive learning theories, alignment of curriculum, and assessment of learning outcomes. While embracing these principles helps support the central principles, it is possible that alternatives could offer similar outcomes.

% \subsection{Focus on Important Aspects} % (fold)
% \label{sub:focus_on_important_aspects}

\pref{itm:focus} encourages a focus on communicating only key concepts, and providing students with access to details which they can use as needed. 

% Reserve communication for core concepts -- lay foundations

% % subsection focus_on_important_aspects (end)

% \subsection{High Expectations} % (fold)
% \label{sub:high_expectations}

% % subsection high_expectations (end)

% \subsection{Support Students} % (fold)
% \label{sub:support_students}

% % subsection support_students (end)

% \subsection{Trust and Empower Students} % (fold)
% \label{sub:trust_and_empower_students}

% % subsection trust_and_empower_students (end)

% \subsection{Be Agile and Willing to Change} % (fold)
% \label{sub:be_agile_and_willing_to_change}

Balancing weekly assessment tasks with staff and student workloads is a challenging tasks, and results in the need to adjust teaching and learning activities in response to feedback received during delivery. This can be seen by the changes to teaching and learning activities between the iterations outlined in \cref{cha:evaluation}.  


% % subsection be_agile_and_willing_to_change (end)

% \subsection{Embed Reflective Practice} % (fold)
% \label{sub:embed_reflective_practice}

% subsection embed_reflective_practice (end)

% subsection supporting_principles (end)


\subsection{Principles related to ``What'' we teach} % (fold)
\label{sub:principles_related_to_}

% subsection principles_related_to_ (end)

% \subsection{Focus Strategy on a Programming Paradigm} % (fold)
% \label{sub:focus_strategy_on_a_programming_paradigm}

% % subsection focus_strategy_on_a_programming_paradigm (end)

% \subsection{Focus on Concepts} % (fold)
% \label{sub:focus_on_concepts}

% % subsection focus_on_concepts (end)

% \subsection{Use Tools Appropriately} % (fold)
% \label{sub:use_tools_appropriately}

% % subsection use_tools_appropriately (end)


% Basically discuss the impact of removing each of the principles from CH3 individually. Reinforce the notion that they work together. 

% What from the Approach can change?
% - Portfolio Assessment?
% - Delivery Approach?
% - Guidelines? (CH4)



Could you do it differently?

Which principles could be left out? Are they all critical?

Do a principle by principle analysis

No frequent formative feedback = heavy workload to assess portfolios (a little each week, or lots at the end) little each week as added benefits of improved learning, lots at the end ???

% section approach_and_principles_in_review (end)

\section{Risk Assessment of Approach} % (fold)
\label{sec:risk_assessment_of_approach}



\subsection{Surface Learning and Plagiarism} % (fold)
\label{sub:surface_learning_and_plagiarism}

Risks:
\begin{itemize}[noitemsep,nolistsep]
	\item Plagiarism of core tasks.
	\item Details are missed by rapidly assessing portfolios.
\end{itemize}

% subsection surface_learning_and_plagiarism (end)

\subsection{Staff and Student Workloads} % (fold)
\label{sub:staff_and_student_workloads}

% subsection staff_and_student_workloads (end)

% section risk_assessment_of_approach (end)






\subsection{Learning from Agile Software Development Practices} % (fold)
\label{sub:learning_from_agile_software_development_practices}

% subsection learning_from_agile_software_development_practices (end)

\subsection{Concept-based Programming} % (fold)
\label{sub:concept_based_programming}

% subsection concept_based_programming (end)

\subsection{Changes Experienced} % (fold)
\label{sub:changes_experienced}

- Widening of perspective - support them in a wider range of activities
- Setting students free, no constraints
- No upper bound
- Supportive environment
- Interaction with students

% subsection changes_experienced (end)


% section reflection_on_findings_and_experience (end)
\clearpage
\section{Challenges for Wider Adoption} % (fold)
\label{sec:challenges_for_wider_adoption}

\subsection{It is easier to be a ``Sage on the stage'' than a ``Guide by the side''} % (fold)
\label{sub:adopting_constructive_learning_theories}

A shift from a primarily objectivist view of education, with educators as the ``sage on the stage'', to one centred on constructive learning theories, where educators are a ``guide by the side'', requires a significant conceptual change. Educators need to move away from delivering material with the goal of ``knowledge transfer'', and consider how activities are likely to assist students with the construction of their own knowledge. This requires a conceptual change, and moves the focus from a teacher-centred environment to a student-centred environment where it is what the students do that \emph{really} counts.

This change is particularly challenging. Knowledge transfer is simple: get a number of people in a room and tell them what they need to know. Guiding students in the construction of their own knowledge seems like a much greater challenge. Accepting the students central role in constructing their own knowledge, means rethinking old strategies, and looking for new ways to engage students with the material. 

Moving from teacher-centred teaching and learning activities, to activities that aim to actively engage students requires a trust in student motivation, as teaching staff relinquish some control over these activities. With teacher-centred activities, such as lectures, teaching staff have complete control of the material. Teaching staff determine the content, pace, and method of delivery. As more student-centred activities are adopted there is a need to incorporate greater input from students, and as a result some level of control is lost. 


There is also a greater need for staff to be experts in the areas they teach,

With knowledge transmission, well scripted lectures can be delivered by teaching staff with a range of abilities -- as long as they can understand the intention of the person who created the slides. Notes on the slides become reminders of ``what to say'', lowering the need for expert teaching staff who understand the concepts and can 

understand students understanding - diagnose root cause of misunderstanding 

and address misconception

This shift in emphasis also requires extra time and attention. Delivering a lecture, and leaving students to do the hard work of trying to understand what was communicated, requires less effort than when teaching staff aim to actively support students in developing their understanding. These additional efforts require recognition by administrators in appropriate allocations of staff workloads, if quality learning is a real goal of higher education.

Once a change in view

Mistakes, or omissions, become a case for  


 

 The ``sage'' is in control, whereas a ``guide'' must at least stick with the students.


\subsection{Releasing control of assessment} % (fold)
\label{sub:releasing_control_of_assessment}

The teacher is in control.

When knowledge transfer is the focus, it becomes easy to lay the blame for learning failures on the students. The knowledge was ``delivered'' to them, so it is their fault if they didn't understand it. When education is more as providing guidance to students, then 

Teaching staff can to  ``I told them...'' so if they then fail to include 





% subsection releasing_control_of_assessment (end)


% subsection adopting_constructive_learning_theories (end)

% section challenges_for_wider_adoption (end)

General applicability of approach - Object about objects first (referenced from Ch5)

Discuss issues with admin processes.

Paradigm shift in OOP
- issue or not?

\section{Formative and Summative Assessment} % (fold)
\label{sec:formative_and_summative_assessment}

Mixed assessment

% section formative_and_summative_assessment (end)

\section{What is a Portfolio?} % (fold)
\label{sec:what_is_a_portfolio_}

Too vague... many report using portfolios but other than Biggs original work none have used it in the same way.


Mixed mode assessment - assignments + portfolio, portfolio + exam.
= mixed messages

Flexible education
- not factory model


% section what_is_a_portfolio_ (end)

\section{Rebellious Robert and Selfless Susan} % (fold)
\label{sec:rebellious_robert_and_selfless_susan}

Approach to learning to program requires understanding... Marton:2005 indicates that education systems are not focused on this and as a result neither do our students... so they fail.

Approach to learning is still the number one issue, portfolio appears to work well for intrinsically motivated students ... but not so well for those who want to avoid the work (no silver bullet)

% section rebellious_robert_and_selfless_susan (end)

\section{Engaging Students in Introductory Programming} % (fold)
\label{sec:engaging_students_in_introductory_programming}


\citet{Guzdial:2005} non-majors

% section engaging_students_in_introductory_programming (end)

\section{Resources} % (fold)
\label{sec:resources}

Short/specific podcasts had a longer life

Concept podcasts were not as successful... more resources?

Separation of activities and resources? - successfull

% section resources (end)

% \subsection{Related work on general education principles} % (fold)
% \label{ssub:related_work_on_education_principles}

% %
% % JG - I am not sure about this section in terms of fit & content...
% %
% % Maybe could incorporate in the principles sections???
% %
% %


% In discussing how to improve undergraduate education, \citet{Chickering:1987} listed seven principles for good practice in undergraduate education. These are practices that:
% \begin{enumerate}[noitemsep,nolistsep]
% 	\item Encourages contact between students and faculty.
% 	\item Develops reciprocity and cooperation among students
% 	\item Encourages active learning
% 	\item Gives prompt feedback
% 	\item Emphasizes time on task
% 	\item Communicates high expectations
% 	\item Respects diverse talents and ways of learning
% \end{enumerate}

% The following list states how each of the principles from \citet{Chickering:1987} are integrated with the principles underlying this work. 
% \begin{itemize}[noitemsep,nolistsep]
% 	\item The strong emphasis on frequent formative feedback should be used to help encourage contact between students and faculty. (\Pref{itm:formative}, \Pref{itm:support})
% 	\item This same formative process should also be harnessed to encourage sharing and cooperation amongst students. (\Pref{itm:formative},\Pref{itm:support},\Pref{itm:theory_y})
% 	\item The central nature of the students in constructing their knowledge necessitates an approach that encourages active learning. (\Pref{itm:construct},\Pref{itm:theory_y})
% 	\item The formative feedback process needs to ensure that work is returned promptly to students, ensuring they receive the feedback while it is still relevant. (\Pref{itm:formative})
% 	\item Communicating high expectations is included directly in our principles. (\Pref{itm:expectations})
% 	\item Assessment and teaching and learning activities need to be flexible, enabling different styles of learning and to engage with students diverse talents. (\Pref{itm:support})
% \end{itemize}





\section{Future Work} % (fold)
\label{sec:future_work}

Structured Literature review - interesting to note referencing in the field

Formally evaluate the time taken

Compare the reliability of assessment


% section future_work (end)

% chapter discussion (end)