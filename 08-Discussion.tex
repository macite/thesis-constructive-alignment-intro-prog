%!TEX root = Constructive Alignment for Introductory Programming.tex

\chapter{Discussion} % (fold)
\label{cha:discussion}


Chapters \ref{cha:guiding_principles} to \ref{cha:evaluation} have presented twelve principles, and their application in the creation and evaluation of a student centred teaching and learning environment based upon \citet{Biggs:1996c} work on constructive alignment. This chapter starts with a reflection


\section{Reflection on Findings and Experience} % (fold)
\label{sec:reflection_on_findings_and_experience}

\subsection{General Applicability of Constructive Alignment} % (fold)
\label{sub:general_applicability_of_constructive_alignment}

Biggs' original proposal of constructive alignment concluded with the following question:

\begin{quote}
	``Can the principle of constructive alignment be generalised from the context of in-service teacher education?'' \citet{Biggs:1996c}
\end{quote}

The work in this thesis indicates that the same student-centred teaching and learning environment can be achieved through the application of the principles outlined in \cref{cha:guiding_principles}. These principles underpin the approach to constructive alignment described in \cref{cha:approach}, which together with the principles guided the design, development, and delivery of the units described in \cref{cha:example_impl}. The results? A supportive, student-centred, teaching and learning environment in which, to use the words of \citet{Biggs:2007} (p.51), students consistently ``\emph{stun}'' teaching staff by the ``\emph{rich and exciting}'' work they demonstrate in their portfolios.

As outlined in \cite{Biggs:1996c}, the model of constructive alignment makes intuitive sense, and comes together as a whole when:
\begin{enumerate}[noitemsep,nolistsep]
	\item Teaching staff are clear about the \emph{intended learning outcomes}.
	\item Assessment criteria are provided to indicate how these outcomes can be met at various levels of achievement, forming a hierarchy from barely satisfactory to most acceptable.
	\item Students are required to perform activities that are likely to elicit the required understandings.
	\item Students provide evidence that their learning has matched the stated outcomes.
\end{enumerate}

\cref{cha:approach} demonstrated how the guiding principles described in \cref{cha:guiding_principles} can be applied to create an approach to teaching introductory programming that encapsulated all of these aspects. The processes described started with the clear expression of intended learning outcomes, with the development of assessment criteria providing the required performance objectives required for different grade outcomes. The development of teaching and learning activities aimed to provide students with tasks likely to engage them in activities that will enable them to construct appropriate understandings, and produce evidence they can include to demonstrate their newly gained knowledge. This evidence could then be collected together and presented in student portfolios as a means of demonstrating how the stated objects had been met.

Therefore, the approach presented in \cref{cha:approach}, along with the example implementation discussed in Chapters \ref{cha:example_impl} to \ref{cha:evaluation}, support Biggs' claim that constructive alignment can be generalised to a range of educational contexts. 

% subsection general_applicability_of_constructive_alignement (end)

\subsection{Beyond Introductory Programming} % (fold)
\label{sub:beyond_introductory_programming}

Can the approach presented in \cref{cha:approach} be used beyond the context of introductory programming?


The use of the approach from \cref{cha:approach} would be particularly well suited to assess learning outcomes for units that include significant use of group work, such as with team-based final year capstone projects. In these unit students work as part of a team, with obvious challenges in assessing the learning outcomes from individual students as final work products are a team effort. Using the approach from \cref{cha:approach} it would be possible to create a learning environment for these units in which:
\begin{itemize}[noitemsep,nolistsep]
	\item Intended learning outcomes capture the required technical and teamwork skills and understandings students needed to demonstrate to successfully complete the unit.
	\item Assessment criteria indicate how these outcomes needs to be demonstrated in order to achieve different grade outcomes.
	\item Students engage in teamwork activities, which are likely to elicit the required outcomes.
	\item Each students collects evidence that they have met all of the intended learning outcomes, aligns their evidence in a Learning Summary Report, and presents this for assessment.
\end{itemize}

In this way, each student's grade would reflect how well they, as individuals, had met the intended learning outcomes. Creating such a scheme would require the embodiment of all principles stated in \cref{cha:guiding_principles}, particularly the need to trust and empower students in their learning (\Pref{itm:theory_y}). 

% subsection beyond_introductory_programming (end)





\subsection{Learning from Agile Software Development Practices} % (fold)
\label{sub:learning_from_agile_software_development_practices}

% subsection learning_from_agile_software_development_practices (end)

\subsection{Concept-based Programming} % (fold)
\label{sub:concept_based_programming}

% subsection concept_based_programming (end)

\subsection{Changes Experienced} % (fold)
\label{sub:changes_experienced}

- Widening of perspective - support them in a wider range of activities
- Setting students free, no constraints
- No upper bound
- Supportive environment
- Interaction with students

% subsection changes_experienced (end)


% section reflection_on_findings_and_experience (end)

\section{Challenges for Wider Adoption} % (fold)
\label{sec:challenges_for_wider_adoption}

\subsection{It is easier to be a ``Sage on the stage'' than it is to be a ``Guide by the side''} % (fold)
\label{sub:adopting_constructive_learning_theories}

A shift from a primarily objectivist view of education, with educators as the ``sage on the stage'', to one centred on constructive learning theories, where educators are a ``guide by the side'', requires a significant conceptual change. Educators need to move away from delivering material with the goal of ``knowledge transfer'', and consider how activities are likely to assist students with the construction of their own knowledge. It is no longer what educator ``say'', as it is what students do that \emph{really} counts.

Knowledge transfer is simple: get a number of people in a room and tell them what they need to know. Guiding students in the construction of their own knowledge seems like a much greater challenge. Accepting the students central role in constructing their own knowledge, means rethinking old strategies, and looking for new ways to engage students with the material.

This shift in emphasis also requires extra time and attention. Delivering a lecture, and leaving students to do the hard work of trying to understand what was communicated, requires less work than when teaching staff aim to actively support students in developing their understanding.  


Mistakes, or omissions, become a case for  


 ``I told them...'' so if they then fail to include 

 The ``sage'' is in control, whereas a ``guide'' must at least stick with the students.


\subsection{Releasing control of assessment} % (fold)
\label{sub:releasing_control_of_assessment}

The teacher is in control.


When knowledge transfer is the focus, it becomes easy to lay the blame the students for learning failures. The knowledge was ``delivered'' to them, so it is their fault if they didn't understand it.  


% subsection releasing_control_of_assessment (end)


% subsection adopting_constructive_learning_theories (end)

% section challenges_for_wider_adoption (end)

General applicability of approach - Object about objects first (referenced from Ch5)

Discuss issues with admin processes.

Paradigm shift in OOP
- issue or not?

\section{Formative and Summative Assessment} % (fold)
\label{sec:formative_and_summative_assessment}

% section formative_and_summative_assessment (end)

\section{What is a Portfolio?} % (fold)
\label{sec:what_is_a_portfolio_}

Too vague... many report using portfolios but other than Biggs original work none have used it in the same way.


Mixed mode assessment - assignments + portfolio, portfolio + exam.
= mixed messages

Flexible education
- not factory model


% section what_is_a_portfolio_ (end)

\section{Rebellious Robert and Selfless Susan} % (fold)
\label{sec:rebellious_robert_and_selfless_susan}

Approach to learning to program requires understanding... Marton:2005 indicates that education systems are not focused on this and as a result neither do our students... so they fail.

Approach to learning is still the number one issue, portfolio appears to work well for intrinsically motivated students ... but not so well for those who want to avoid the work (no silver bullet)

% section rebellious_robert_and_selfless_susan (end)

\section{Engaging Students in Introductory Programming} % (fold)
\label{sec:engaging_students_in_introductory_programming}


\citet{Guzdial:2005} non-majors

% section engaging_students_in_introductory_programming (end)

\section{Resources} % (fold)
\label{sec:resources}

Short/specific podcasts had a longer life

Concept podcasts were not as successful... more resources?

Separation of activities and resources? - successfull

% section resources (end)

% \subsection{Related work on general education principles} % (fold)
% \label{ssub:related_work_on_education_principles}

% %
% % JG - I am not sure about this section in terms of fit & content...
% %
% % Maybe could incorporate in the principles sections???
% %
% %


% In discussing how to improve undergraduate education, \citet{Chickering:1987} listed seven principles for good practice in undergraduate education. These are practices that:
% \begin{enumerate}[noitemsep,nolistsep]
% 	\item Encourages contact between students and faculty.
% 	\item Develops reciprocity and cooperation among students
% 	\item Encourages active learning
% 	\item Gives prompt feedback
% 	\item Emphasizes time on task
% 	\item Communicates high expectations
% 	\item Respects diverse talents and ways of learning
% \end{enumerate}

% The following list states how each of the principles from \citet{Chickering:1987} are integrated with the principles underlying this work. 
% \begin{itemize}[noitemsep,nolistsep]
% 	\item The strong emphasis on frequent formative feedback should be used to help encourage contact between students and faculty. (\Pref{itm:formative}, \Pref{itm:support})
% 	\item This same formative process should also be harnessed to encourage sharing and cooperation amongst students. (\Pref{itm:formative},\Pref{itm:support},\Pref{itm:theory_y})
% 	\item The central nature of the students in constructing their knowledge necessitates an approach that encourages active learning. (\Pref{itm:construct},\Pref{itm:theory_y})
% 	\item The formative feedback process needs to ensure that work is returned promptly to students, ensuring they receive the feedback while it is still relevant. (\Pref{itm:formative})
% 	\item Communicating high expectations is included directly in our principles. (\Pref{itm:expectations})
% 	\item Assessment and teaching and learning activities need to be flexible, enabling different styles of learning and to engage with students diverse talents. (\Pref{itm:support})
% \end{itemize}





\section{Future Work} % (fold)
\label{sec:future_work}

structured Literature review - interesting to note referencing in the field

% section future_work (end)

% chapter discussion (end)