%!TEX root = Constructive Alignment for Introductory Programming.tex

\chapter{Discussion} % (fold)
\label{cha:discussion}

\section{Formative and Summative Assessment} % (fold)
\label{sec:formative_and_summative_assessment}

% section formative_and_summative_assessment (end)

\section{What is a Portfolio?} % (fold)
\label{sec:what_is_a_portfolio_}

Too vague... many report using portfolios but other than Biggs original work none have used it in the same way.


Mixed mode assessment - assignments + portfolio, portfolio + exam.
= mixed messages

% section what_is_a_portfolio_ (end)

\section{Rebellious Robert and Selfless Susan} % (fold)
\label{sec:rebellious_robert_and_selfless_susan}

Approach to learning to program requires understanding... Marton:2005 indicates that education systems are not focused on this and as a result neither do our students... so they fail.

Approach to learning is still the number one issue, portfolio appears to work well for intrinsically motivated students ... but not so well for those who want to avoid the work (no silver bullet)

% section rebellious_robert_and_selfless_susan (end)

\section{Engaging Students in Introductory Programming} % (fold)
\label{sec:engaging_students_in_introductory_programming}


\citet{Guzdial:2005} non-majors

% section engaging_students_in_introductory_programming (end)

\section{Resources} % (fold)
\label{sec:resources}

Short/specific podcasts had a longer life

Concept podcasts were not as successful... 

Separation of activities and resources? - successfull

% section resources (end)

% \subsection{Related work on general education principles} % (fold)
% \label{ssub:related_work_on_education_principles}

% %
% % JG - I am not sure about this section in terms of fit & content...
% %
% % Maybe could incorporate in the principles sections???
% %
% %


% In discussing how to improve undergraduate education, \citet{Chickering:1987} listed seven principles for good practice in undergraduate education. These are practices that:
% \begin{enumerate}[noitemsep,nolistsep]
% 	\item Encourages contact between students and faculty.
% 	\item Develops reciprocity and cooperation among students
% 	\item Encourages active learning
% 	\item Gives prompt feedback
% 	\item Emphasizes time on task
% 	\item Communicates high expectations
% 	\item Respects diverse talents and ways of learning
% \end{enumerate}

% The following list states how each of the principles from \citet{Chickering:1987} are integrated with the principles underlying this work. 
% \begin{itemize}[noitemsep,nolistsep]
% 	\item The strong emphasis on frequent formative feedback should be used to help encourage contact between students and faculty. (\Pref{itm:formative}, \Pref{itm:support})
% 	\item This same formative process should also be harnessed to encourage sharing and cooperation amongst students. (\Pref{itm:formative},\Pref{itm:support},\Pref{itm:theory_y})
% 	\item The central nature of the students in constructing their knowledge necessitates an approach that encourages active learning. (\Pref{itm:construct},\Pref{itm:theory_y})
% 	\item The formative feedback process needs to ensure that work is returned promptly to students, ensuring they receive the feedback while it is still relevant. (\Pref{itm:formative})
% 	\item Communicating high expectations is included directly in our principles. (\Pref{itm:expectations})
% 	\item Assessment and teaching and learning activities need to be flexible, enabling different styles of learning and to engage with students diverse talents. (\Pref{itm:support})
% \end{itemize}


Supports Biggs claims regarding this


\section{Future Work} % (fold)
\label{sec:future_work}

structured Literature review - interesting to note referencing in the field

% section future_work (end)

% chapter discussion (end)