%!TEX root = Constructive Alignment for Introductory Programming.tex

\chapter{Approaches to Constructive Alignment} % (fold)
\label{cha:background}

\section{Constructive Alignment} % (fold)
\label{sec:constructive_alignment}

\subsection{Constructivism} % (fold)
\label{sub:constructivism}

In designing teaching and learning contexts, educators use some form of theory of teaching and learning to guide their decision making.

theory of teaching and learning

espoused theory \cite{Argyris:1976}

\emph{objectivist}

\emph{constructivism} and \emph{phenomenography}







\cite{Montessori:1946}

Constructivism is a theory of knowledge that focuses on the active role of the learner in constructing their own understanding. Dating back to \citet{Piaget:1950} constructivism exists in several forms: cognitive, individual, postmodern, radical and social constructivism \cite{Phillips:1995,Steffe:1995}. Each of these forms of constructivism has various implications for teaching and learning.









In his original paper on constructive alignment \citet{Biggs:1996c} adopted constructivism as a framework to help guide decision making in all facets of teaching and learning. Constructivism was chosen over phenomenography 

practical concerns


For this work we are interested in adapting \emph{constructive learning theories}

 taking practical aspects from constructivism in general and focusing on what the student does, as suggested by .


In proposing constructive alignment, 

Disconnect between theory in use and espoused theory: \cite{Phillips:2005}


Engage with and expand experience \cite{Dewey:1960} exploration, thinking and reflection

% subsection constructivism (end)

\subsection{Aligned Curriculum} % (fold)
\label{sub:aligned_curriculum}

% subsection aligned_curriculum (end)

\subsection{Reported Applications of Constructive Alignment} % (fold)
\label{sub:reported_applications_of_constructive_alignment}

% subsection reported_applications_of_constructive_alignment (end)

% section constructive_alignment (end)

\section{Portfolio Assessment} % (fold)
\label{sec:portfolio_assessment}

% section portfolio_assessment (end)




% chapter background (end)