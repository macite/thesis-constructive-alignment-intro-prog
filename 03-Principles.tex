%!TEX root = Constructive Alignment for Introductory Programming.tex

\chapter{Guiding Principles} % (fold)
\label{cha:guiding_principles}

\graphicspath{{Figures/CAApproach/}}

This chapter describes twelve principles that underlie our model for delivering constructive aligned introductory programming. These principles act as guidelines for decision making, and in many ways underpin the model as a unit's intended learning outcomes underpin constructively aligned teaching. Each aspect of the model, the associated curriculum, teaching and learning activities, and assessment tasks are aligned with one or more of these principles. 

The principles cover both \emph{how} the teaching and learning environment should operate, and \emph{what} should be taught. Originally derived from constructive alignment, the \emph{how} principles centre on constructivism and aligned curriculum. In relation to \emph{what} should be taught, the principles draw upon computing education literature and our own experiences as educators and software developers. 

Reflective practice played an important part in these principles, and both sets of principles have developed over the course of this research. This chapter presents the current working principles we use to guide the development and delivery of introductory programming. While most were present throughout the research, their individual emphasis and relationships have developed through our reflective practice. 

The chapter first outlines the principles related to \emph{how} we aim to teach introductory programming, and then presents the principles related to \emph{what} we aim to teach.



\section{Principles for how the environment should operate} % (fold)
\label{sub:principles_for_how_the_environment_should_operate}

The first nine principles we present relate to \emph{how} we want the teaching and learning environment to operate. Interventions that impact on the learning environment, also referred to as teaching or academic environment, have the potential to positively influence student learning outcomes \cite{Trigwell:1991}. Learning environments have been found to influence students' approach to learning \cite{Entwistle:1990,Entwistle:1991,Kember:2007} and perceptions of teaching environments have been shown to directly, and indirectly, influence learning outcomes \cite{Meyer:1990,Lizzio:2002}.

The principles we present aim to create a positive learning environment for students, one that is demanding of them but supports and rewards their efforts to understand the concepts being presented. In the following list we outline the principles we used to guide our decisions on \emph{how} the teaching and learning environment should operate. These principles are general and could, therefore, be applied to a range of teaching and learning contexts and topic domains. The following list outlines these principles.

\begin{enumerate}
	\item Recognise that students construct knowledge in response to activity.
	\item Align activities and assessment to intended learning outcomes.
	\item Aim to assess learning outcomes not learning pace or product outcomes.
	\item Focus on depth of understanding over breadth of coverage.
	\item Have high expectations of students.
	\item Actively support student efforts.
	\item Trust and empower students to control their own learning.
	\item Be agile and willing to change.
	\item Embed reflective practice in all aspects.
\end{enumerate}

% \subsection{Relationships between Principles} % (fold)
% \label{ssub:relationships_between_principles}

Individually each principle has its own intrinsic value but they are designed to work together. Together, each principle interacts with the other principles to create a productive student centred learning environment.

\begin{figure}[htbp]
	\centering
	\includegraphics[width=\textwidth]{HowPrinciples}
	\caption{Key interaction between proposed principles.}
	\label{fig:how_principles}
\end{figure}

\fref{fig:how_principles} shows the key interactions between these principles. The students active construction of knowledge is central, with various aspects of this being supported by the other principles. Each principle is discussed in detail in the following sections, with the various relationships being discussed along with associated literature.

% section relationships_between_principles (end)
\bigskip
\subsection{Recognise that students construct knowledge in response to activity} % (fold)
\label{ssub:ideas_adopted_from_constructivism}

Decisions about curriculum, teaching and learning activities and assessment tasks are all guided by the educator's theory of teaching and learning \cite{Argyris:1976,Ramsden:1992}. While constructivism is often the espoused theory \citet{Phillips:2005} indicates this has not transitioned to common education practice, resulting in a ``dissonance'' between the elements of effective learning and the characteristics of typical university learning environments. This is symptomatic of the disconnect between educators espoused theory and their theory-in-use \cite{Argyris:1976}. To successfully implement constructive alignment it will, therefore, be important to adopt the key aspects from constructivism outlined by \citet{Biggs:1996c}, \citet{Biggs:1997} and in \citet{Biggs:2007}. By consciously attempting to adopt constructivism as our theory-in-use we aim to create a educational setting which is ``in harmony'' with the principles of constructive alignment.

Central to all forms of constructivism is the principle that learning is an active process requiring the leaner to construct their own understanding through individual and social activity \cite{Biggs:1996c,Cunningham:1996,Duffy:1992,Glasersfeld:1989,Steffe:1995}. During the development of the model presented in the next chapter we actively worked to embed the following aspects of constructivism in our theory-in-use.

\begin{itemize}
	\item Knowledge is constructed, not transmitted via communication alone.
	\item Teaching involves creating a context in which learns are able to construct appropriate cognitive models through individual and social activities.
	\item Errors in understanding are opportunities for further learning, these help indicate the students' current level of development and can be used to guide future learning activities.
\end{itemize}

\citet{Biggs:1996c} reason for adopting constructivism as a central philosophy was due to its emphasis on the students active role in constructing their own knowledge. When taken to an extreme this results in approaches that rely upon students building their own understanding from first principles, such as in discovery learning \cite{Bruner:1961}. These approaches are often promoted in constructivist writings, such as in \citet{Glasersfeld:1989} and \citet{Cunningham:1996}. The unstructured nature of these teaching and learning environments have received strong criticism. \citet{Anderson:1998} criticises constructive learning theories when ``pursued to unproductive extremes'', as in the case with discovery learning. \citet{Mayer:2004} argues against discovery learning, instead suggesting that constructivist views of education may be better served through cognitive activity, instructional guidance, and curricular focus. Furthermore, \citet{Kirschner:2006} argue against discovery learning indicating that in highly complex environment, such as with software development, free exploration may generate a heavy workload and detrimentally affect learning.

While we do take on board the central role of the learner in constructing their knowledge, we want to avoid detrimental aspects associated with taking these ideas to their extreme. As a result, we temper constructivism with certain practical details, an approach we feel is in line with the principles of constructive alignment. These details include the following:

\begin{itemize}
	\item Communication remains a valuable tool in helping shape the learning context.
	\item Guided instruction is valuable and ensures student activity is likely to be productive.
	\item Deliberate practice provides students with opportunities to engage with principles in action.
\end{itemize}

% section ideas_adopted_from_constructivism (end)

\subsection{Align activities and assessment to intended learning outcomes} % (fold)
\label{ssub:align_activities_and_assessment_to_intended_learning_outcomes_}

The second pillar of constructive alignment is aligned curriculum. The impact of aligning teaching and learning activities and assessment tasks was highlighted by \citet{Cohen:1987}. In discussing instructional alignment \citet{Cohen:1987} reported that the effect sizes based on achievement tests was up to four times greater than in non-aligned instruction. In proposing constructive alignment \citet{Biggs:1996c} aligns teaching and learning activities and assessment tasks to intended learning outcomes, weaving a ``web of consistency'' to optimise the likelyhood of students engaging appropriately with learning activities \cite{Biggs:1999}. 

In order to implement constructive alignment this ``web of consistency'' must also be achieved. The alignment of activities and assessment to intended learning outcomes will be critical for our model. As shown in \fref{fig:how_principles}, this alignment is seen as supporting student construction of knowledge. By aligning teaching and learning activities to the intended learning outcomes we ensure that students are constructing the required understanding. Similarly, by aligning assessment with these same intended learning outcomes we ensure that students are adequately prepared for this assessment and that the assessment is evaluating unit goals. 

% subsection align_activities_and_assessment_to_intended_learning_outcomes_ (end)


\subsection{Aim to assess learning outcomes not learning pace or product outcomes} % (fold)
\label{ssub:aim_to_assess_learning_outcomes_not_learning_pace_or_product_outcomes_}

Assessment in education is often seen as serving one of two purposes: supporting learning, or evaluating outcomes. These two purposes have come to be known as \emph{formative} and \emph{summative}, as proposed by \citet{Scriven:1967} and \citet{Bloom:1969}. To best support student construction of knowledge we want to maximise formative assessment. Using frequent formative feedback during the semester to aid students in developing appropriate understanding, and delaying summative assessment until after unit delivery.

To avoid assessing students pace of learning 100\% of each student's grade will be determined from the summative assessment. This ensures students are free to use the formative assessment to develop their understanding. As this work will not have marks attached students can highlight their misunderstandings and uncertainties without losing marks. We want to encourage students to use these formative tasks to demonstrate what they \emph{do not} know, as much as what they \emph{do}. 

Our assessment should, as much as possible, focus on providing feedback on student understanding and ability to meet the intended learning outcomes. We want to focus on more than just the ``product'' outcomes from the teaching and learning activities. Assessment tasks need to include aspects that require students to articulate their current understanding of concepts. This can then be used to help determine the students current level of understanding, and errors evident in this work provides opportunities for students to learn from their mistakes. 

%Traditionally programming units focus on code writing activities, and assessment then focuses on evaluating the programs produced. \citet{Lister:2004} augmented this with code reading and tracing activities to help better assess student understanding. We want to take this work further and assess, formatively and summatively, students ability to \emph{describe} and \emph{explain} underlying principles, code structures, and development processes. This will mean the code itself will be insufficient, and additional teaching and learning activities will need to get students to describe and explain aspects of software development.


This principle relates to both students construction of knowledge and to the alignment of activities and assessment, as shown in \fref{fig:how_principles}. Formative assessment of learning outcomes during delivery helps students in the construction of knowledge, providing opportunities to learn from their mistakes without fear of losing marks. These formative tasks also help both staff and students with the alignment of teaching and learning activities and assessment tasks. Staff can use these misunderstandings to help guide students individually, and to change or adjust teaching and learning activities where needed. For students, this ongoing focus on articulation of understanding and receiving feedback will ensure they are suitably prepared to demonstrate how they have met all of the intended learning outcomes in the final summative assessment.

The summative assessment also contributes to both students construction of knowledge and to the alignment of activities and assessment. For final unit grades students will need to demonstrate how their understanding aligns with the units intended learning outcomes. The assessment needs then to aim to assess the level of understanding demonstrated, in effect aiming to evaluate how suitable the students level of understanding is at the end of the unit.

Our position on the use of assessment in this way has significant support in the education research literature with the value of feedback being widely reported. In proposing the use of formative assessment in education \citet{Bloom:1969} indicates that an assessment item can play both formative and summative roles, though he suggests that in this case the formative assessment will be less effective. \citet{Ramsden:1992} indicates that of all items on the Course Evaluation Questionnaire \cite{Ramsden:1991} the one that most clearly distinguished between the best and worst courses related to the provision of helpful feedback. \citet{Black:1998} show that substantial learning gains can be achieved by innovations designed to strengthen frequent feedback students receive. Furthermore, \citet{Black:1998} report that students pay more careful attention to feedback when there are no associated marks. In discussing assessment for learning \citet{Brown:2004} states that ``Formative feedback is critical'' and that ``feedback must be at the heart of the process'' if we are to make assessment an integral part of learning.

% \citet{Mattick:2007} identified the perceived lack of feedback as a barrier to creating a high quality learning environment for undergraduate medical students. 

The experience of \citet{Smith:2005}, however, indicates that shifting to formative assessment requires more than just removing the marks. \citet{Smith:2005} reported significantly worse results for the group of early secondary student who received only formative feedback. In discussing their results, \citet{Smith:2005} indicate that in this case comments were not often constructive, were misunderstood by students, and were not integrated back into the teaching and learning context. \citet{William:2006} indicates that to be formative, outcomes of the assessment must be used make adjustments to help students better meet learning needs. It appears that in the case examined by \citet{Smith:2005} the assessment remained primarily summative in nature with marks being replaced by comments. To be effective, our formative feedback must be communicated effectively to students and provide them with clear means of addressing any shortcomings.

\citet{Gibbs:2004} lists ten conditions under which assessment assists with student learning, shown below. To help address the issues identified by \citet{Smith:2005} we need to ensure that all of these conditions can be met.
\begin{enumerate}
	\item Sufficient assessed tasks are provided for students to capture sufficient study time.
	\item These tasks are engaged with by students, orienting them to allocate appropriate amounts of time and effort to the most important aspects of the course.
	\item Tackling the assessed task engages students in productive learning activity of an appropriate kind.
	\item Sufficient feedback is provided, both often enough and in enough detail.
	\item The feedback focuses on students' performance, on their learning and on actions under the students' control, rather than on the students themselves and on their characteristics.
	\item The feedback is timely in that it is received by students while it still matters to them and in time for them to pay attention to further learning or receive further assistance.
	\item Feedback is appropriate to the purpose of the assignment and to its criteria for success.
	\item Feedback is appropriate, in relation to students' understanding of what they are supposed to be doing.
	\item Feedback is received and attended to.
	\item Feedback is acted upon by the student.
\end{enumerate}

% subsection aim_to_assess_learning_outcomes_not_learning_pace_or_product_outcomes_ (end)

\subsection{Focus on depth of understanding over breadth of coverage} % (fold)
\label{ssub:focus_on_depth_of_understanding_over_breadth_of_coverage_}

With limited resources, primarily time, the classic depth-vs-breadth trade off needs to be considered. Given that we have fixed time, we can either focus on providing a breadth of coverage, or a depth of understanding. As programming is central to the discipline of computing \cite{McGettrick:2005} it will be important to focus on building depth of understanding, over a breadth of coverage.

The study conducted by \citet{Schwartz:2009} found that high school students who reported studying a major topic in depth earned higher grades in college than those who reported covering no topics in depth. If this can translate to undergraduate computing education, then a depth of understanding in programming may help students succeed with other computing units. Given the strong correlation generally observed between programming skills and other computing skills \cite{McGettrick:2005} this is likely.

Depth over breadth will help focus the teaching and learning activities and assessment tasks, and ensure that they align to sufficiently deep cognitive levels. As the tasks define what students will do, this will in turn help ensure that students are developing appropriately deep knowledge in relation to the intended learning outcomes.

% subsection focus_on_depth_of_understanding_over_breadth_of_coverage_ (end)

\subsection{Have high expectations of students} % (fold)
\label{ssub:have_high_expectations_of_students_}

Believing in students' potential and communicating high expectations will be important in the development of the model. In listing their principles for good undergraduate education \citet{Chickering:1987} include communicating high expectations as one of their seven principles. \citet{Chickering:1987} state ``expect more and you will get more'' and indicate that high expectations are important for everyone, from those who are poorly prepared or unwilling to exert themselves to those who are bright and motivated. Believing in students' potential is also key to the approach presented by \citet{Soetanto:2003,Soetanto:2012}. Soetanto's approach aims to improving students' discipline, confidence and belief in their potential, and units delivered with this approach have gained in popularity despite their technical difficulty and being delivery in a foreign language (English).

By having high expectations of our students we hope to build student confidence and get them to aspire to excellence. These expectations will then require students to work hard throughout the unit's delivery, providing encouragement to spend sufficient time on the teaching and learning activities. This will require both time and energy from students, which should improve outcomes: as stated by \citet{Chickering:1987} ``time plus energy equals learning''.

% subsection have_high_expectations_of_students_ (end)

\subsection{Actively support student efforts} % (fold)
\label{ssub:actively_support_student_efforts}

Both \citet{Chickering:1987} and \citet{Soetanto:2003,Soetanto:2012} indicate high expectations should also apply to teaching staff. If we are to reasonably expect a lot from our students, they should expect a lot from us. To help students achieve the required understanding we need to actively support their efforts. This will need to extend beyond providing formative feedback, to providing active support throughout the process. Given the technical nature of introductory programming and the exacting nature of compiler, students are going to face numerous challenges. Learning to program \textbf{is} hard, so we must make every effort to support our students.

% subsection actively_support_student_efforts (end)

\subsection{Trust and empower students to control their own learning} % (fold)
\label{ssub:trust_and_empower_students_to_control_their_own_learning}

Student motivation has a significant impact on learning. In terms of strategies for improving student motivation, \citet{McGregor:1960} work on motivational strategies in business provide some insights into similar strategies that could be applied to education.  In his work on personnel management, \citet{McGregor:1960} identified two means of categorising how managers perceived their employees: named Theory X and Theory Y. Traditionally businesses were seen to use coercion or persuasion as a strategy to motivate employees to achieve required levels of productivity. These strategies are used when managers adopt the view that employees do not want to work and cannot be trusted, \citet{McGregor:1960} named this understanding of human motivation as Theory X. In contrast, Theory Y assumes that, given the right conditions, people want to work, that they can be trusted and will do their best work when they are.

While applied to business organisation management, these two views can also be applied to an educational setting. \citet{Biggs:2007} argues for the adoption of Theory Y climates to get the most out of students. This is supported by \citet{Markwell:2004} who categorises Theory X teaching environments as ones where teachers assume that:

\begin{itemize}
	\item Students have little desire to learn new material.
	\item Students are inherently lazy and will attempt to get the material dumbed-down; the teacher must use a controlling environment to force students to learn and prevent cheating.
	\item Students prefer to be directed and do not want to be responsible for their own learning.
	\item The teacher must act as the source of information and actively transmit it to the students.
	\item Many students are not capable of learning the necessary material and can be expected to earn a low grade.
\end{itemize}

\noindent This is in contrast to list provided by \citet{Markwell:2004} for Theory Y environments where teachers assume that:

\begin{itemize}
	\item Learning is as natural to students as play or rest.
	\item Students are not lazy; threats of diminished grades are not necessary to motivate students.
	\item The self-satisfaction from learning is sufficient to commit students to achieving the educational objectives.
	\item Students will naturally accept responsibility for learning.
	\item Imagination, ingenuity, and creativity are widely distributed within the student population and will be willingly applied to the learning process.
	\item The intellectual potential of most students are being only partially utilized in the classroom.
\end{itemize}

In order to achieve many of the principles listed here it will be necessary for us to adopt a predominantly Theory Y stance. The formative nature of the assessment tasks together with the high expectations will both require a level of trust in students that cannot be achieved with a predominantly Theory X stance. High expectations of students is a natural repercussion of Theory Y, and enhancing motivation in this way should help students in the construction of their knowledge.

% subsection trust_and_empower_students_to_control_their_own_learning (end)

\subsection{Embed reflective practice in all aspects} % (fold)
\label{ssub:embed_reflective_practice_in_all_aspects}

In education the idea of reflective practice is to periodically look back at our teaching, and consider how things can be improved. The foundations of this idea can be traced back \citet{Dewey:1933}, though reflective practice was originally proposed by \citet{Schon:1983}. \citet{Farrell:2007,Farrell:2008} identify two forms of reflective teaching practice: a strong form and a weak form. In its weak form, reflective practice involves informal evaluation of various aspects of professional practice. \citet{Farrell:2008} likens this to no more than thoughtful practice. The alternative strong form of reflective practice involves systematic reflect on teaching and taking responsibility for teaching and learning activities. To ensure ongoing improvements, the model we propose needs to incorporate reflective practice. We propose to user reflective teaching ``hand-in-hand with critical self-examination and reflection as a basis for decision making, planning, and action'' \cite{Richards:1994}. % (p.ix)

From a practical perspective, we can use common misconceptions identified in formative feedback to update delivery during the semester. After the semester, we can use students' results, both in terms of grade distributions and quality of evidence demonstrated in the final summative assessment, to suggest changes for following iterations. Reflections on teaching should be shared amongst all teaching staff related to the unit to encourage all staff to reflect on their practice, and to enable them to learn from each other.

Reflective practice also needs to be encouraged in the student body. Students undertaking units taught with this approach will graduate and move into professional practice. Engaging them with reflective practice throughout their education will help ensure they are adequately equipped to engage in lifelong learning \cite{Field:2006}. The active incorporation of frequent formative feedback will provide a direct means of encouraging students to reflect on their work throughout the delivery of the unit. To further encourage reflection, their summative assessment should include some reflective aspects where they can reflect on what they have achieved in the unit. 

Reflection underpins all of the principles presented. Students engage in reflection as a tool to help them construct their knowledge, and the formative feedback enables student and staff reflections. Staff will reflect on teaching and learning activities, their alignment to learning outcomes, and the depth and breadth of coverage. Reflections will enable us to realistically manage expectations, the support we offer students, and to help us balance trust with mechanisms to avoid exploitation. Even the nature and composition of these principles have been the focus of our ongoing reflective practice.

% subsection embed_reflective_practice_in_all_aspects (end)


\subsection{Be agile and willing to change} % (fold)
\label{ssub:be_agile_and_willing_to_change}

For reflective practice to actively enhance our teaching we must embrace change, focusing on aspects that will deliver the most value for students for the effort required. As software developers, this emphasis on delivering value by focusing on the things that matter most was reminiscent of the agile software development principles \cite{Martin:2003}. The Agile Manifesto \cite{Beck:2001} laid down several key values: 

\begin{itemize}
	\item Individuals and interactions over processes and tools
	\item Working software over comprehensive documentation
	\item Customer collaboration over contract negotiation
	\item Responding to change over following a plan
\end{itemize}

\noindent To realise all of these principles it will be necessary to adopt similar values in our teaching, valuing things that help students construct their knowledge over other activities.

To help manage change effectively we view the environment as consisting of an underlying strategy, teaching and learning resources, and teaching and learning activities as shown in \fref{fig:strategy}. The strategy relates to the overall plan for delivering the unit content and informs the development of the intended learning outcomes. The central role of the intended learning outcomes means that the overall strategy should not change unless significant issues are identified in the approach overall. Teaching and learning resources can then be separated from the teaching and learning activities. In this way we can create reusable resources that are independent of the activities that are used in, meaning that activities can be adjusted more freely to better help direct student efforts. The details of this are summarised in the following list.

\begin{figure}[htbp]
	\centering
	\includegraphics[width=0.8\textwidth]{StrategyResourcesActivities}
	\caption{Relationship between strategy resources and activities.}
	\label{fig:strategy}
\end{figure}

\begin{itemize}
	\item Develop an overall \emph{strategy} for delivering the unit content.
	\begin{itemize}
		\item Derive intended learning outcomes from the strategy.
		\item Use strategy to focus other teaching and learning resources and activities.
		\item Avoid changing strategy, unless approach has significant issues.
	\end{itemize}

	\item Create teaching and learning \emph{resources}.
	\begin{itemize}
		\item Deliver material following the direction from the overall strategy.
		\item Make resources generic and self contained.
		\item See as providing a supporting role to the teaching and learning activities.
		\item Invest in initial development, actively reuse and enhance over time.
		\item Improve integration with delivery material as resources prove to be effective.
	\end{itemize}

	\item Design teaching and learning \emph{activities}.
	\begin{itemize}
		\item Use these to focus student activity.
		\item Actively review each semester, and rework based on reflection.
	\end{itemize}
\end{itemize} 

% subsection be_agile_and_willing_to_change (end)


\subsection{Related Work on General Education Principles} % (fold)
\label{ssub:related_work_on_education_principles}

In discussing how to improve undergraduate education \citet{Chickering:1987} lists the following seven principles for good practice in undergraduate education:
\begin{enumerate}
	\item Encourages Contact Between Students and Faculty.
	\item Develops Reciprocity and Cooperation Among Students
	\item Encourages Active Learning
	\item Gives Prompt Feedback
	\item Emphasizes Time on Task
	\item Communicates High Expectations
	\item Respects Diverse Talents and Ways of Learning
\end{enumerate}


The following list outlines how each of the principles from \citet{Chickering:1987} was integrated with the principles underlying this work. 

\begin{itemize}
	\item The strong emphasis on frequent formative feedback should be used to help encourage contact between students and faculty.
	\item This same formative process should also be harnessed to encourage sharing and cooperation amongst students.
	\item The central nature of the students in constructing their knowledge necessitates an approach that encourages active learning.
	\item The formative feedback process needs to ensure that work is returned promptly to students, ensuring they receive the feedback while it is still relevant.
	\item Communicating high expectations is included directly in our principles.
	\item Assessment and teaching and learning activities need to be flexible, enabling different styles of learning and to engage with students diverse talents.
\end{itemize}

% For outcomes based education, 

% \citet{Killen:2000} lists the 

% subsection related_work_on_education_principles (end)


\subsection{Summary of principles on how to teach} % (fold)
\label{ssub:summary_of_principles_on_how_to_teach}

In this section we have presented the nine principles by which we will guide decisions making for how to teaching introductory programming. Each principle is backed by a range of education theories, and together the principles should enable us to create an environment that is demanding but supportive, focused on students building knowledge, agile, constantly improving through reflections, and accepting of various strategies and pace of learning.

The next section details the principles related to \emph{what} we are going to teach.

% subsection summary_of_principles_on_how_to_teach (end)


% section principles_for_how_the_environment_should_operate (end)

\clearpage
\section{Principles to guide what we should teach} % (fold)
\label{sec:principles_to_guide_what_we_should_cover}

Principles specific to teaching introductory programming are presented in the following list. While the general principles helped shape the overall teaching and learning environment, these principles helped shape specifics in the curriculum, activities, and assessment tasks.

\begin{itemize}
	\item Set the strategy and structure learning around a programming paradigm.
	\item Focus on programming concepts, not language syntax.
	\item Use programming languages as they were designed to be used.
	% \item Program code is not, by itself, a suitable means of measuring learning outcomes.
\end{itemize}

Each of these principles is expanded upon in the following sections, and linked to associated research.


\subsection{Set the strategy and structure learning around programming a paradigm} % (fold)
\label{ssub:set_the_strategy_and_structure_learning_around_programming_a_paradigm}

The programming paradigm chosen for introductory programming will have a large impact on the way the unit is taught and, therefore, the unit's intended learning outcomes. To ensure that the foundations of the unit do not need to change, a paradigm needs to be chosen and this will then form the core of the overall strategy for the unit.

Which programming paradigm should be taught first is a popular topic in computing education research. The range of approaches include imperative programming first such as with \citet{Koffman:1988a}, \citet{Howe:2004} presented a components-first approach, and \citet{Bennedsen:2004} use a model-first approach. \citet{Cooper:2003} suggest a course working with graphics prior to introductory programming to help students with problem solving skills. With the predominant role of objects in industry, a common trend has been to move from imperative-first approaches to objects-first approaches, an approach that has ``failed'' according to \citet{Astrachan:2005} and \citet{Reges:2006}. While \citet{Ehlert:2009} report no significant difference between approaches using objects-first and objects-later, though their later work \cite{Ehlert:2010} indicated the objects-later approach had a greater comfort level for students. \citet{Robins:2003} include the imperative-first and objects-first approaches as one of their four themes from the literature on learning and teaching introductory programming. \citet{Lister:2006a} provide an in depth look at the research perspectives on the topic of objects-first, but in general there is no consensus on which approach should be taken.

The design of our introductory programming units will need to choose which \emph{x}-first approach will be used. The main decision seems to be between the imperative-first, objects-first or functional-first approaches. Whichever approach is taken, this will guide which concepts are covered in the unit and the order in which these can be tackled.

% subsection set_the_strategy_and_structure_learning_around_programming_a_paradigm (end)

\subsection{Focus on programming concepts} % (fold)
\label{sub:focus_on_programming_concepts}

There are various views of programming in the literature. On perspective views programming from a mathematical basis \cite{Denning:1989,Dijkstra:1989,Hoare:1969}, others see it as an exercise in problem solving \cite{Palumbo:1990}, modelling concepts \cite{Bennedsen:2004}, though most textbooks approach the topic through language syntax and feature \cite{Robins:2003}. In this work we avoid the standard textbook approach, and instead we will focus on programming concepts.

A similar idea was expressed in \citet{Goldman:2004}. The \emph{concept based} approach of \citet{Goldman:2004} introduced students to a number of ``big ideas'' related to software development using the JPie interactive programming environment. Our approach differs in that we focus on the ``small ideas'' that programs are build upon. In this way we will build depth across these ideas rather than breadth, in line with our general principles on how we will deliver this.

Topics such as variables, procedures, control flow, etcetera will be approached as a concept. This is similar to the model based approach of \citet{Bennedsen:2004}, but applied to procedural programming concepts. By focusing on concepts we aim to provide students with reasons why the various programming features should be used, when different abstractions should be used, and help them develop means of picturing these abstract ideas. This concept based approach can be applied to both procedural programming concepts and object oriented programming concepts.

\citet{Winslow:1996} comparison of expert and novice programmers indicate that expert programmers tend to ``abstract from a particular language to the general concept''. By focusing on the concepts first we hope to instil similar ideas in the students. Students will be encouraged to create \emph{conceptual programs}, these can then be mapped to code using the programming language syntax rules. Focusing first on the concepts should encourage a depth of understanding, with students going beyond the surface syntax used and thinking conceptually about what they are trying to achieve.

Programming concepts are tightly interrelated, and yet we need to provide a sequence of activities that will introduce students without overloading them initially. In designing these activities we aimed to ensure that we could introduce concepts without having to rely upon ``magic'', limiting the cases where students had to do some thing without being able to reason why with the concepts covered so far. Given the imperative-first approach, this suits a bottom up delivery. 

By achieving this we aim to enable students to explore language features. At each stage in the process students should have sufficient concepts to be able to understand the programs we ask them to create. To extend the students further we can then ask them to experiment with features, using the concepts we have covered to create programs they are interested in. This should enhance student motivation, and ensure they spend sufficient time on the task.

The focus on programming concepts will mean that we need to:
\begin{itemize}
	\item Introduce programming concepts incrementally.
	\item Provide students with time to put concepts into practice.
	\item See syntax as a means to an end, not an end in itself.
	\item Avoid using language features before concepts that can explain their use.
	\item Map concepts to code using programming language grammars.
\end{itemize} 

% section principles_to_guide_what_we_should_cover (end)

\subsection{Use programming languages as they were designed to be used} % (fold)
\label{ssub:use_programming_languages_as_they_were_designed_to_be_used}

While we agree with the ``back to basics'' approach of \citet{Reges:2006}, we want to emphasise the using programming languages in the way they were designed to be used. Our emphasis on programming concepts has relegated language to a secondary role, and one that can be changed by learning new syntax rules. If this is achieved changing between languages is less problematic, and therefore we should not use a language for its industry relevance. Rather, we should choose a language based on the concepts it supports, its available across computing platforms, and the level of support it offers novices.

Programming language choice for introductory programming is as popular in the Computing Education Research literature as the choice of programming paradigm. The following list covers some of the many papers on which programming language to use in teaching introductory programming, ordered by year it shows the shift from procedural languages to object oriented languages.

\begin{itemize}
	\item \citet{Koffman:1988} argues for Modula-2 over Pascal, PL/1 and Ada, 
	\item \citet{Mody:1991} argues against C, and C++ for its lack of coherence, simplicity, understandability and implementability. 

	\item \citet{Roberts:1993} discusses Stanford's shift from Pascal to C, addressing common issues with C by providing libraries to encapsulate complex features, emphasising procedural and modular abstractions, and focusing on the discipline of software engineering.

	\item \citet{Brilliant:1996} discuss programming paradigm and language selection for introductory programming. They conclude that there is no clear advantage to starting with either procedural programming, object-oriented programming, or functional programming paradigms. In terms of language they discuss moving from Pascal to Ada, C, C++, or Scheme.

	\item \citet{Boszormenyi:1998} argues for Modula-3 over Java discussing features that are useful to be taught in introductory programming but missing from Java.

	\item \citet{Howell:2003} claim that through structured labs, comprehensive grade sheets, in class grading, and frequent feedback any language can be used.

	\item \citet{Gupta:2004} suggests the first language should strike a balance between being easy to grasp and supporting advanced concepts needed for later units.

	\item \citet{Kelleher:2005} provides a detailed examination of a wide range of programming languages used for teaching introductory programming. This work discusses various efforts to help make programming more accessible for novices.

	\item \citet{Bishop:2006} presents the pros and cons from their experiences using the C\# programming language, and provide some recommendations for those looking at using the language.

	\item \citet{Mannila:2006} argue for the use of Python due to its simplicity, with examples comparing Python to Java.

	\item \citet{Mannila:2006a} provides a set of criteria for comparing language features for introductory programming, with Eiffel, Java and Python achieving the highest scores.

	\item \citet{Pendergast:2006} provides a reflection on teaching introductory programming with Java over a number of years, highlighting some of the issues encountered and mechanisms to avoid these.

	\item \citet{Maloney:2010} describes the Scratch programming environment, a visual programming language designed primarily for students aged between 8 and 16.

	\item \citet{Anik:2011} uses the Analytic Network Process methodology to help guide the decision of which language should be used first. The results ranked Java above other languages.
\end{itemize}

Underlying many of these papers is the idea that switching language is difficult. For example, \citet{Brilliant:1996} indicates that teaching multiple languages increases the overhead necessary to cover language details and peculiarities. By having a focus on programming concepts over language syntax we aim to tackle this problem from a different angle. In our previous teaching we noticed that students tended to focus and cling to syntax, shifting language was a major effort as it was syntax they had learnt. Students felt they were ``Java programmers'' or ``C/C++ programmers'', they did not see that they were learning something far more important, they were not learning Java they were learning to program. This focus becomes evident when you examine the title of many introductory programming units. Introductory programming units like ``Introduction to Programming with C'' and ``Software Development in Java'' give the impression that the language is important. 

In choosing a language for our \emph{concept-based} approach to introductory programming we were guided by the following principles:
\begin{itemize}
	\item Remember we are not teaching a language, we are teaching students to program.
	\item Do not teach the language explicitly, teach students to learn languages themselves.
	\item Choose a language, or languages, that support the concepts that need to be covered.
	\item Ensure support for multiple operating systems, enabling students to use their preferred platform.
	\item See multiple languages as an important part of the learning, not an unnecessary overhead.
	\begin{itemize}
		\item Using multiple languages will encourage students to focus on concepts, not syntax.
		\item By encountering multiple languages students will see different languages have different strengths and weaknesses.
		\item Encourage students to see they are learning to program, not one language.
		\item Foster the attitude that language is a choice, students should not feel constrained to one language and should be open to possibilities other languages offer.
	\end{itemize}
\end{itemize}

One aspect that is different from other work is the importance of the use of the language in industry. If the concept based approach is successful students will be able to quickly develop skills in industry relevant languages in later units. The first units can focus on using languages that best meet educational requirements, while later units can cover language specific details and peculiarities briefly knowing students have an understanding of underlying concepts and the ability to learning new languages themselves. 

% \cite{Koffman:1988,Roberts:1993,Brilliant:1996}
% Arguments against Pascal 
% \begin{enumerate}
% 	\item Missing language features such as open arrays, function pointers, and string manipulation.
% 	\item Not widely used beyond introductory programming.
% 	\item Lack of data abstraction and information hiding.
% \end{enumerate}


% subsection use_programming_languages_as_they_were_designed_to_be_used_ (end)


\subsection{Summary of principles for what to teach} % (fold)
\label{ssub:summary_of_principles_for_what_to_teach}

This section has proposed the use of a \emph{concept-based} approach to teaching introductory programming. This approach will focus on teaching programming concepts directly and use programming language grammars to help students map these concepts to code. The aim of this approach is to help students focus on the concepts, while providing them with tools they can use to learn the programming language. If this approach is successful the choice of which programming language is used is less important than which paradigm. 

% subsection summary_of_principles_for_what_to_teach (end)

\clearpage
\section{Summary of Guiding Principles} % (fold)
\label{sec:summary_of_guiding_principles}

This chapter has outlined twelve principles related to both \emph{how} and \emph{what} we aim to teach in our constructively aligned introductory programming. The nine principles for how 

\begin{itemize}
	\item Nine principles describe \emph{how} the teaching and learning environment should operate:
	\begin{itemize}
		\item Recognise that students construct knowledge in response to activity.
		\item Align activities and assessment to intended learning outcomes.
		\item Aim to assess learning outcomes not learning pace or product outcomes.
		\item Focus on depth of understanding over breadth of coverage.
		\item Have high expectations of students.
		\item Actively support student efforts.
		\item Trust and empower students to control their own learning.
		\item Be agile and willing to change.
		\item Embed reflective practice in all aspects.
	\end{itemize}
	\item Three principles help guide \emph{what} we aim to teach:
	\begin{itemize}
		\item Set the strategy and structure learning around a programming paradigm.
		\item Focus on programming concepts, not language syntax.
		\item Use programming languages as they were designed to be used.
		% \item Program code is not, by itself, a suitable means of measuring learning outcomes.
	\end{itemize}
\end{itemize}

The next chapter outlines our model for constructively aligned introductory programming. The model presented was created through the application of these principles, and its ability to embody these principles is discussed.

% section summary_of_guiding_principles (end)

% chapter guiding_principles (end)